\section{Motivating Applications for Face Recognition}
The ability of humans to quickly and accurately recognize each-other by sight
is one of the foundations of offline social interaction.  As digital devices
(both mobile and embedded in our infrastructure) increase in importance for our
daily lives, so does the incentive to share with them our capacity for
automatic visual recognition.  While many applications of face recognition are
controversial due to privacy concerns, there are far more potential
applications where the advantages are clear.  Credit fraud could be
significantly reduced if automated teller machines and cash registers were able
to differentiate customers from thieves carrying stolen wallets.  Theft of
devices could be reduced if they were only responsive to their owners.
Customized user interfaces and access to data could propagate between devices.
Mechanical door locks could be replaced with systems that are simultaneously
more accurate and more convenient.  The primary allure of
automated face recognition is its potential to make the initiation of
authenticated interaction with a machine as natural as making eye contact with
another human.

While many of the applications of face recognition could be addressed using
other biometrics such as fingerprint recognition, iris recognition, face
recognition has the potential of being much less intrusive to users of the
system; it is non-contact, and the user need not take any action beyond turning
their head towards the device they want to interact with (even iris recognition
typically requires the user to carefully position their head and keep their
eyes open).  

It is important to maintain a clear distinction between face {\em recognition}
and face {\em verification}.  In face recognition the task is to both determine
if the probe subject is one of the gallery subjects, and if so, to accurately
determine the identity of the probe subject within the gallery subjects.  In
face verification, the task is only to determine if the probe subject is the
same as a specific user of claimed identity.  Using these definitions, face
verification is a special case of face recognition when there is only a single
gallery user.  Examples of automated face recognition are automated checking of electronic
passports and automated login to laptops and single user computer systems.  While some
of the techniques described in this work are also applicable to automated face
verification, the emphasis is on the more challenging face recognition problem.

Face recognition applications (and research) can be roughly categorized by the
demanded recognition rate, and by the quality of available data.  Low-stakes
applications such as online image search and family photo album organization
(e.g.\ Google Picassa, Microsoft Photo Gallery, and Apple iPhoto) have been
tackled successfully in large part since they are useful at low recognition
rates when combined with a good user interface.  The detection of attempts to register
for state identification twice is
another such application; for the system to be useful it is sufficient to narrow
down the gallery to a subset small enough to be checked by a human
investigator.

Another category of face recognition applications involves recognition using
many (often uncooperative) users using restricted gallery images.  This
application includes terrorist watchlist applications, applications in mass
surveillance and tracking, and electronic advertisements capable of recognizing
people.  While this category is the most thoroughly studied, it is largely
unsolved due to the combination of a need to operate with many gallery users,
as well as the need to be able to operate with restricted gallery images.  
Law enforcement requires compatibility with old mugshots for the gallery,
and often the probe image may not be frontal in surveillance images.

This thesis argues that there is a large and under-studied category of
recognition applications where very high recognition accuracy is desired, but
the users in the gallery are still allies of the system rather than
adversaries.  These applications include access control for secure facilities
(e.g., prisons, office buildings), computer systems, automobiles, or automatic
teller machines, where controlled gallery images can be obtained in advance.
These applications are very interesting due to their potential sociological
impact.  Since the gallery subjects are allies, rather than opponents, of the
recognition system, it is feasible to carefully control the acquisition of the
gallery images. 

Many classic and contemporary face recognition algorithms work well on public
data sets, but degrade sharply when they are used in a real recognition system.
This is mostly due to the difficulty of simultaneously handling variations in
illumination, image misalignment, and occlusion in the test image. We consider
a scenario where the training images are well controlled, and test images are
only loosely controlled.  This thesis describes a conceptually simple face
recognition system that achieves a high degree of robustness and stability to
illumination variation, image misalignment, and partial occlusion.  First, well
registered training images taken under many illumination directions are
captured using a novel projector-based acquisition system.  The recognition
system then uses tools from sparse representation to align a test face image to
a set of frontal training images.  To better handle severe occlusions an
extension to the algorithm is described that makes use of the knowledge that
occluded pixels tend to be spatially correlated.  Due to the use of multiple
face images as features and as the non-smooth nature of the optimization
problems, these techniques have far greater computational requirements than
techniques that extract low-dimensional features.  Several custom $\ell_1$
solvers are presented that achieve faster convergence on face data than general
solvers.  optimized implementations for modern parallel computing architectures
are investigated in order to a build a system capable of perform highly
accurate and robust recognition while remaining fast enough for use in access
control system.  Optimized parallel implementations for contemporary CPU and
GPU hardware are demonstrated to achieve near real-time face recognition for
acces control applications with hundreds of gallery users.





\section{Previous Work} Very few recognition systems specifically target
applications where many well- controlled training images are available.  Of
these, the classical holistic subspace-based face recognition methods
\cite{Turk1991-CVPR,Belhumeur1997-PAMI} are well known for their speed and
simplicity, as well as for their natural extension to linear illumination
models.  However, their performance has been shown to be extremely brittle not
only to alignment variation, but to even minor occlusions caused by, say, a
wisp of hair, a blinked eye, or mouth that is slightly open. 

One of the logistical difficulties that has been holding face recognition
research back is that, even with cooperative subjects, it is very difficult to
collect sufficient data to achieve meaningful recognition rates.  It takes a
lot of resources to simultaneously build custom hardware for a training image
acquisition system, manage the capturing of images of over a hundred test
subjects, and still have the resources to implement an advanced recognition
algorithm.  This fact has contributed to a pattern where published face
recognition research is conducted almost exclusively on public data sets.
While public face databases play an important role in allowing researchers to
compare the performance of their algorithms, relying on them exclusively for
research prevents the researcher from tightly integrating their algorithm with
their training image acquisition system.  This thesis demonstrates that, in
order to achieve the very high recognition rates that are needed for access
control applications, tight integration between the training image acquisition
system and the recognition system is a practical necessity.  In particular,
many published algorithms implicitly make photometric assumptions that are
(often unnecessarily) violated by the data sets they run on.  

\section{Introduction to $\ell_1$ minimization and sparse representation based classification}
%
$\ell_1$-minimization has received much attention in recent years due to
important applications in compressive sensing \cite{BrucksteinA2007} and sparse
representation \cite{WrightJ2010-PIEEE}.  
In general, $\ell_1$-minimization can refer to any minimization problem involving the 
$ell_1$-norm (sum of absolute values, noted as $||\cdot||_1$) of a vector of expressions involving the optimization
variables. However, in the context of this thesis, we will be primarily concerned with
the class of optimization problems that minimize the $\ell_1$-norm of a vector that
is affine in the optimization variables, under constraints that are also affine in the optimization variables.
One common sparse representation formulation finds the minimum $\ell_1$-norm solution to an
underdetermined linear system $\bb=A\xx$:
%
\begin{equation} \min \|\xx\|_1\quad \mbox{ subj. to }\quad \bb = A\xx.
\label{eq:l1min} \end{equation}
%
It is now well established that, under certain conditions
\cite{CandesE2005-IT_1,DonohoD2004}, the minimum $\ell_1$-norm solution is also
the \emph{sparsest} solution to the system \eqref{eq:l1min}.

In addition to numerous other applications, $\ell_1$-minimization has been recently used to reformulate
image-based face recognition as a sparse representation problem
\cite{WrightJ2009-PAMI}.  If we stack the $m$ pixels of the $n_i$ training images of $K$ subject
classes into the columns of matrices $(A_1\in\Re^{m\times n_1}, \cdots, A_K\in\Re^{m\times n_K})$, combine
the matrices into a larger matrix $A = [A_1, \cdots, A_K]\in\Re^{m\times n}$, and arrange the pixels of a new
query image into a vector $\bb\in\Re^m$, \emph{sparsity-based
classification} (SBC) solves the following minimization problem:
\begin{equation}
\min_{\xx, \ee} \| \xx \|_1 + \|\ee\|_1 \quad \subj \quad \bb = A \xx + \ee.
\label{eq:l1min_denoise}
\end{equation}
If the sparsest solutions for $\x$ and $\e$ are recovered, $\ee$ provides a
means to compensate for pixels that are corrupted due to occlusion of some part of the query
image, and the dominant nonzero coefficients of $\xx$ reveal the membership of
$\bb$ based on the training image labels associated with $A$. 
If $A$ contains images of each subject taken under different illuminations, 
$A_i\x_i$ acts as a linear illumination model for the test image $\bb$.  The motivation
for this illumination model will be further discussed in Chapter \ref{chap_pipeline}, Section \ref{sec:illumination}.
SRC has demonstrated striking recognition performance
despite severe occlusion or corruption by solving a simple convex program.  For
this reason, the final recognition stage of the recognition pipeline will consist of
SRC, and the iterative face alignment stage that precedes it will be based on
similar techniques.

\section{Document Structure}
%
Chapter \ref{chap:pipeline} is devoted to presenting a complete face
recognition pipeline, which was developed in collaboration with John Wright,
Zihan Zhou, Arvind Ganesh, Hossein Mobahi, and Yi Ma.  This pipeline was
presented at the IEEE Conference on Computer Vision and Pattern Recognition
\cite{WagnerA2009-CVPR}.  An improved version of this recognition pipeline that
is both faster and more accurate was was published in \cite{WagnerA2011-PAMI}.
%
Chapter \ref{chap:iccv} presents an extension of the algorithm that better
handles image occlusions by making use of the knowledge that occluded image
pixels tend to be adjacent to each other, modeling the occlusion distribution
with a Markov Random Field.  This work was presented at the IEEE International
Conference on Computer vision \cite{ZhouZ2009}.
%
Chapter \ref{chap:minimization} discusses the application of several numerical
optimization techniques to the core minimization problems required by the face
recognition pipeline.
%
Chapter \ref{chap:parallel} presents the optimized parallel implementations of
the core numerical solvers, as well as the alignment algorithm, on highly
concurrent multicore CPU and on GPU hardware.
%
Finally, Chapter \ref{chap:future} discusses a variety of ideas for future
improvements to the recognition system, and some of the remaining
implementation challenges that will be required for the system to be ready for
commercial application.
 
