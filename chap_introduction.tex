
One of the most important skills humans possess is the ability to quickly
recognize each other by sight.   The ability to rapidly identify each other
gives us very important information we can use to interact with the people in
the world around us.  Face recognition is such a useful skill that automating
this ability has been one of the major topics of computer vision research.
Unfortunately, it turns out that human faces are rather difficult to recognize,
primarily for reasons that are inherent in the imaging process.  Obstacles to
high performance face recognition include an (initial) absence of information
about the mapping between a person's face and an image, variations in nature of
the light illuminating a person's face, and (unlabeled) portions of a person's
face that may not have even been captured on camera due to occlusion.  

One of the difficulties that has been holding face recognition back is that is
is very difficult to collect sufficient data to compute meaningful recognition
rates.  It takes a lot of resources to simultaneously build custom hardware for
a training image acquisition system, manage the capturing of images of over a
hundred test subjects, and still have time to implement an advanced recognition
algorithm.  While public face databases play an important role in allowing
researchers to compare the performance of their algorithms, relying on them
exclusively for research prevents the researcher from tightly integrating their
algorithm with their training image acquisition system.  In order to achieve
the very high recognition rates that are needed for access control
applications, tight integration between the training image acquisition system
and the recognition system is necessary.  In particular, many published
algorithms make photometric assumptions that are (often unnecessarily) violated
by the data sets they run on.

This thesis proposal demonstrates that a tightly integrated training image
acquisition system and recognition algorithm is capable of significantly
improving face recognition performance.  A key part of the system is exploiting
the an important property of the imaging process:  there in a linear mapping
between the space of illuminations of an object, and the space of images of
that object taken under the same pose.  This makes it possible to effectively
model the testing image as a linear superposition of a large (and well chosen)
set of training images.  This idea is certainly not new; indeed it has been in
use for face recognition for roughly two decades, \cite{Turk1991-CVPR}.
However, traditional algorithms that rely on this property of the image
formation process have tended to perform very badly in the face of occlusions
and when highly quality training images are unavailable.  the recent 

Chapter \ref{chap:cvpr} is devoted to presenting a complete face recognition
pipeline that I have built in collaboration with John Wright, Zihan Zhou,
Arvind Ganesh, and Yi Ma.  This body of work has been accepted for publication
in the IEEE Conference on Computer Vision and Pattern Recognition
\cite{Wagner2009-CVPR}.  This main contributions of this work are a novel
system for training image acquisition and an automatic image alignment system
that turn the algorithm of \cite{Wright2009-PAMI} into a complete recognition
system.  Chapter \ref{chap:iccv} presents an extension of the algorithm that
better handles image occlusions by making use of the knowledge that occluded
image pixels tend to be adjacent to each other, modeling the occlusion
distribution with a Markov Random Field.  Finally, Chapter \ref{chap:proposed}
discusses a variety of ideas for future improvements to the recognition system.

%Unfortunately, a good recognition algorithm alone is not sufficient to
%construct a practical face recognition system: a linear superposition of the
%training images must be able to represent the test image taken under a
%different lighting condition.  For this purpose I will present a novel
%training image acquisition system that is able to rapidly acquire images of a
%subject under varying illumination.  While many acquisition systems have been
%built for this purpose, they have not been practical for widespread use.  I
%will propose a projector based approach, which can be easily assembled from
%off-the-shelf parts, and which is much more flexible in the variety of
%illuminations it can produce.

%Since general software packages for solving sparse representation problems are
%unable to run in a reasonable amount of time on high dimensional face
%recognition data, I will show how dramatic speed gains can be achieved by
%offloading the data parallel portions of the algorithm onto a massively
%parallel processor.  I will demonstrate a GPU based implementation of a sparse
%solver that can perform recognition quickly enough to be useful for access
%control applications.  Finally, I will demonstrate a complete face recognition
%system that significantly advances the state of the art in terms of its
%ability to efficiently and effectively recognize faces under a variety of
%realistic conditions.


%Most contemporary face recognition algorithms work well under laboratory
%conditions but degrade when tested in less-controlled environments. This is
%mostly due to the difficulty of simultaneously handling variations in
%illumination, alignment, pose, and occlusion.  In this thesis proposal, we
%propose a simple and practical  face recognition system that achieves a high
%degree of  robustness and stability to all these variations. We demonstrate
%how to use tools from sparse representation to align a test face image with a
%set of frontal training images in the presence of significant registration
%error and occlusion. We thoroughly characterize the region of attraction for
%our alignment algorithm on public face datasets such as Multi-PIE.  We further
%study how to obtain a sufficient set of training illuminations for linearly
%interpolating practical lighting conditions. We have implemented a complete
%face recognition system, including a projector-based training acquisition
%system, in order to evaluate how our algorithms work under practical testing
%conditions. We show  that our system can efficiently and effectively recognize
%faces under  a variety of realistic conditions, using only frontal images
%under the proposed illuminations as training.\vspace{-10mm}

%Partially occluded faces are common in many applications of face recognition.
%While algorithms based on sparse representation have demonstrated promising
%results, they achieve their best performance on occlusions that are not
%spatially correlated (i.e.\ random pixel corruption). We show that such
%sparsity-based algorithms can be significantly improved by harnessing prior
%knowledge about the pixel error distribution. We show how a Markov Random
%Field model for spatial continuity of the occlusion can be integrated into the
%computation of a sparse representation of the test image with respect to the
%training images.  Our algorithm efficiently and reliably identifies the
%corrupted regions and excludes them from the sparse representation.  Extensive
%experiments on both laboratory and real-world datasets show that our algorithm
%tolerates much larger fractions and varieties of occlusion than current
%state-of-the-art algorithms. \vspace{0mm}


