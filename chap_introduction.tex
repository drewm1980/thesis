\chapter{Introduction}
\label{chap:introduction}
 
The ability of humans to quickly and accurately recognize each-other by sight
is one of the foundations of offline social interaction.  As digital devices
(both mobile and embedded in our infrastructure) increase in importance for our
daily lives, so does the incentive to share with them our capacity for
automatic visual recognition.  While many applications of face recognition are
controversial due to privacy concerns, there are far more potential
applications where the advantages are clear.  Credit fraud could be
significantly reduced if automated teller machines and cash registers were able
to differentiate customers from thieves carrying stolen wallets.  Theft of
devices could be reduced if they were only responsive to their owners.
Customized user interfaces and access to data could propagate between devices.
Mechanical door locks could be replaced with systems that are simultaneously
more accurate and more convenient.  The primary allure of
automated face recognition is its potential to make the initiation of
authenticated interaction with a machine as natural as making eye contact with
another human.

\section{Positioning of this work} Face recognition applications to date have
fallen into roughly two categories.  Face recognition has recently seen a lot
of success in a family of less-demanding applications such as online image
search and family photo album organization (e.g.\ Google Picassa, Microsoft
Photo Gallery, and Apple iPhoto). At the other end of the tractability spectrum
there are the terrorist watchlist and mass surveillance applications that have
for the most part dominated the field of face recognition research.  However,
there are many face recognition applications that fall roughly between these
extremes, where very high recognition performance is desired, but the users in
the gallery are still allies of the system rather than adversaries.  These
applications include access control for secure facilities (e.g., prisons,
office buildings), computer systems, automobiles, or automatic teller machines,
where controlled gallery images can be obtained in advance.  These applications
are very interesting due to their potential sociological impact.  Since the
gallery subjects are allies, rather than opponents, of the recognition system,
this creates the possibility of carefully controlling the acquisition of the
training data. While the same can be said for other biometrics such as
fingerprints and iris recognition, face recognition has the potential of
working with test data that is much less controlled, allowing the access
control system to be made less intrusive to the users of the system.  The scope
of this thesis is to show how a reliable face recognition system can be built
for this restricted, but still important, scenario.  


Unfortunately, it turns out that human faces are rather difficult to recognize
in images. for reasons that are inherent in the imaging process.  Obstacles to
highly accurate face recognition include an (initial) absence of information
about the mapping between a person's face and an image, variations in nature of
the light illuminating a person's face, and (unlabeled) portions of a person's
face that may not have even been captured on camera due to occlusion.  

\section{Previous Work} Very few recognition systems specifically target
applications where many well- controlled training images are available.  Of
these, the classical holistic subspace-based face recognition methods
\cite{Turk1991-CVPR,Belhumeur1997-PAMI} are well known for their speed and
simplicity, as well as for their natural extension to linear illumination
models.  However, their performance has been shown to be extremely brittle not
only to alignment variation, but to even minor occlusions caused by, say, a
wisp of hair, a blinked eye, or mouth that is slightly open. 

One of the logistical difficulties that has been holding face recognition
research back is that, even with cooperative subjects, it is very difficult to
collect sufficient data to achieve meaningful recognition rates.  It takes a
lot of resources to simultaneously build custom hardware for a training image
acquisition system, manage the capturing of images of over a hundred test
subjects, and still have the resources to implement an advanced recognition
algorithm.  This fact has contributed to a pattern where published face
recognition research is conducted almost exclusively on public data sets.
While public face databases play an important role in allowing researchers to
compare the performance of their algorithms, relying on them exclusively for
research prevents the researcher from tightly integrating their algorithm with
their training image acquisition system.  This thesis demonstrates that, in
order to achieve the very high recognition rates that are needed for access
control applications, tight integration between the training image acquisition
system and the recognition system is a practical necessity.  In particular,
many published algorithms implicitly make photometric assumptions that are
(often unnecessarily) violated by the data sets they run on.  

\section{Introduction to $\ell_1$ minimization and sparse representation based classification}
%
$\ell_1$-minimization has received much attention in recent years due to
important applications in compressive sensing \cite{BrucksteinA2007} and sparse
representation \cite{WrightJ2010-PIEEE}.  
In general, $\ell_1$-minimization can refer to any minimization problem involving the 
$ell_1$-norm (sum of absolute values, noted as $||\cdot||_1$) of a vector of expressions involving the optimization
variables. However, in the context of this thesis, we will be primarily concerned with
the class of optimization problems that minimize the $\ell_1$-norm of a vector that
is affine in the optimization variables, under constraints that are also affine in the optimization variables.
One common sparse representation formulation finds the minimum $\ell_1$-norm solution to an
underdetermined linear system $\bb=A\xx$:
%
\begin{equation} \min \|\xx\|_1\quad \mbox{ subj. to }\quad \bb = A\xx.
\label{eqn:l1min} \end{equation}
%
It is now well established that, under certain conditions
\cite{CandesE2005-IT_1,DonohoD2004}, the minimum $\ell_1$-norm solution is also
the \emph{sparsest} solution to the system \eqref{eqn:l1min}.

In addition to numerous other applications, $\ell_1$-minimization has been recently used to reformulate
image-based face recognition as a sparse representation problem
\cite{WrightJ2009-PAMI}.  If we stack the $m$ pixels of the $n_i$ training images of $K$ subject
classes into the columns of matrices $(A_1\in\Re^{m\times n_1}, \cdots, A_K\in\Re^{m\times n_K})$, combine
the matrices into a larger matrix $A = [A_1, \cdots, A_K]\in\Re^{m\times n}$, and arrange the pixels of a new
query image into a vector $\bb\in\Re^m$, \emph{sparsity-based
classification} (SBC) solves the following minimization problem:
\begin{equation}
\min_{\xx, \ee} \| \xx \|_1 + \|\ee\|_1 \quad \subj \quad \bb = A \xx + \ee.
\label{eqn:l1min_denoise}
\end{equation}
If the sparsest solutions for $\x$ and $\e$ are recovered, $\ee$ provides a
means to compensate for pixels that are corrupted due to occlusion of some part of the query
image, and the dominant nonzero coefficients of $\xx$ reveal the membership of
$\bb$ based on the training image labels associated with $A$. 
If $A$ contains images of each subject taken under different illuminations, 
$A_i\x_i$ acts as a linear illumination model for the test image $\bb$.  The motivation
for this illumination model will be further discussed in Chapter \ref{chap:pipeline}, Section \ref{sec:illumination}.
SRC has demonstrated striking recognition performance
despite severe occlusion or corruption by solving a simple convex program.  For
this reason, the final recognition stage of the recognition pipeline will consist of
SRC, and the iterative face alignment stage that precedes it will be based on
similar techniques.

\section{Document Structure}
%
Chapter \ref{chap:pipeline} is devoted to presenting a complete face
recognition pipeline, which was developed in collaboration with John Wright,
Zihan Zhou, Arvind Ganesh, Hossein Mobahi, and Yi Ma.  This pipeline was
presented at the IEEE Conference on Computer Vision and Pattern Recognition
\cite{WagnerA2009-CVPR}.  An improved version of this recognition pipeline that
is both faster and more accurate was was published in \cite{WagnerA2011-PAMI}.
%
Chapter \ref{chap:markov} presents an extension of the algorithm that better
handles image occlusions by making use of the knowledge that occluded image
pixels tend to be adjacent to each other, modeling the occlusion distribution
with a Markov Random Field.  This work was presented at the IEEE International
Conference on Computer vision \cite{ZhouZ2009}.
%
Chapter \ref{chap:minimization} discusses the application of several numerical
optimization techniques to the core minimization problems required by the face
recognition pipeline.
%
Chapter \ref{chap:parallel} presents the optimized parallel implementations of
the core numerical solvers, as well as the alignment algorithm, on highly
concurrent multicore CPU and on GPU hardware.
%
Finally, Chapter \ref{chap:future} discusses a variety of ideas for future
improvements to the recognition system, and some of the remaining
implementation challenges that will be required for the system to be ready for
commercial application.
 
