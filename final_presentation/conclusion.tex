\section[Conclusion]{Conclusion and Future Work}

\frame{ \frametitle{Conclusion} 
\begin{itemize}
\item Projector-based image acquisition system dramatically simplifies capture of subjects under many illuminations
\item Iterative alignment techniques achieve excellent robustness to alignment variations
\item Proposed promising techniques for improving robustness to severe occlusions 
\item Developed ALM based solvers that dramatically improve scalability of global representation
\item Designed and implemented highly optimized parallel solvers on CPU and GPU architectures
\end{itemize}
}

\frame{ \frametitle{Future Work: Improving pipeline speed} 
\begin{itemize}
\item Improve Recognition Speed
\begin{itemize}
\item On GPU, parallelized ALM implementations are bandwidth limited
\item Want to reduce number of loads of $A$ per iteration, \# of iterations
\item Can afford to spend more FLOPS on any subproblem that fits in cache
\end{itemize}
\item Try to leverage geometry of $\ell_1$ minimization problem:
\begin{itemize}
\item Exact line search only costs $m \log m$ with small constant
\item Expensive part of descent direction computation operates on $n \times n$ subarray
\item Multi-Scale within solver to leverage correlation of adjacent pixels
\end{itemize}
\item Faster training image acquisition 
\begin{itemize}
\item Multiplexed Illuminations can be used to increase SNR or decrease exposure time...
\item ...but require calibrated projectors.
\item Automatically calibrate projectors? 
\end{itemize}
\end{itemize}
}

\frame{ \frametitle{Future Work: Improving recognition performance} 
How can we perform iterative alignment even with severe occlusions?
\begin{itemize}
\item Explore alternative convex norms that encourage contiguous error (i.e. C-norm).
\item Attempt to distinguish between occlusion and misalignment error components.
\item Random initialization techniques [Yang, Shia] are investigating this in face tracking context.
\end{itemize}
Why does global representation (i.e. via SRC) improve imposter rejection so much?
\begin{itemize}
\item LBP achieves better R.R. but SRC has better ROC curve. 
\item CVPR paper from another group (under review) claims explanation for $\ell_2$ case, but result is flawed.
\end{itemize}
}

