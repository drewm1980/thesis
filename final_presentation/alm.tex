\section{Fast ALM Based Algorithms for $\ell_1$-Minimization}
%\frame{\tableofcontents[currentsection, currentsubsection]}
%\subsection{Introduction}

%\frame{ \frametitle{$\ell_1$ Minimization Problems}
%The core of the recognition
%pipeline consisted of solving $\ell_1$ minimization problems with the following two forms:
%\begin{equation}
%\min_{\x, \e} \| \x \|_1 + \|\e\|_1 \quad \subj \quad \bb = A \x + \e.
%\end{equation}
%\begin{equation}
%\min_{\x, \e} \|\e\|_1 \quad \subj \quad \bb = A \x + \e.
%\end{equation}
%}

\frame{ \frametitle{$ell_1$ Minimization Algorithms} 
\begin{itemize}
\item Interior point method converges in small number of steps,  but each
step is very expensive, involves solving linear system and re-ordering data in inner loop.
\item To overcome this, a state-of-the-art
$\ell_1$-minimization solution based on \emph{augmented Lagrangian methods}
(ALM) [BertsekasD2003,YangA2010-ICIP] has been developed.
\item Belongs to family of iterative shrinkage-thresholding methods, contains only
first-order operations in its inner loop.
\end{itemize}
}

%\subsection{Augmented Lagrange Multipier Method}
%\frame{\tableofcontents[currentsection, currentsubsection]}


\frame{\frametitle{ALM Derivation}
The following minimization problem is used in the recognition stage:
\begin{equation}
\min_{\x, \e} \| \x \|_1 + \|\e\|_1 \quad \subj \quad \bb = A \x + \e.
\end{equation}
The corresponding augmented Lagrangian function is:
\begin{equation}
L_\mu (\x,\e,\blambda) = \|\x\|_1 + \|\e\|_1 + \langle \blambda, \bb-A\x - \e \rangle
+ \frac{\mu}{2} \|\bb - A\x - \e\|_2^2,
\end{equation}
where $\blambda$ is the Lagrange multiplier and $\mu > 0$ is a
penalty parameter. 
}

\frame{\frametitle{ALM Derivation}
The ALM method seeks a saddlepoint of $L_\mu
(\x,\e,\blambda)$ by alternating between optimizing with respect
to the primal variables $\x, \e$ and updating the dual variable
$\blambda$, with the other fixed, as follows:
\begin{equation}
\left \{
\begin{array}{lll}
(\x_{k+1},\e_{k+1})  =  \arg\min_{(\x,\e)} \, L_{\mu} (\x,\e,\blambda_k),\\
\blambda_{k+1}  =  \blambda_k + \mu (\bb - A\x_{k+1} - \e_{k+1}). \\
\end{array}
\right .
\label{eqn:alm}
\end{equation}
Although updating $\blambda$ is trivial,
minimizing $L_{\mu} (\x,\e,\blambda_k)$ with respect to both
$\x$ and $\e$ could still be costly. 
}

\frame{\frametitle{ALM Derivation} 
To further reduce the complexity of the
problem, we adopt an approach used in alternates between minimizing 
$L_{\mu} (\x,\e,\blambda_k)$ over $\x$ (with $\e$ fixed) and over $\e$ (with $\x$
fixed). After solving these two subproblems, the Lagrange multiplier $\blambda$
is updated, yielding an iteration of the form:

%\begin{equation}
%%\left \{
%\begin{array}{lll}
%\e_{k+1}  =  \argmin{\e} L_{\mu} (\x_k,\e,\blambda_k) 
%\x_{k+1}  =  \argmin_{\x} \, L_{\mu} (\x,\e_{k+1},\blambda_k),\\
%\blambda_{k+1}  =  \blambda_k + \mu (\bb - A\x_{k+1} - \e_{k+1}).
%\end{array}
%%\right \. 
%\end{equation}
%As the objective function is convex and alternation is between two
%terms, this procedure is guaranteed to converge to a global optimum (see [YangJ2009-pp] and references therein).
}

%In order to discuss the solution to the above subproblems, we
%need to define the following soft-thresholding operator for a
%vector $\x$ and a scalar $\alpha \geq 0$:
%\begin{equation}
%\mathcal{T}(\x,\alpha) = \textup{sign}(\x)\cdot \max \{|\x| - \alpha, 0\},
%\end{equation}
%where all the operations are performed component-wise. It is
%easy to show that the subproblem with respect to $\e$ has a
%closed-form solution given by the soft-thresholding operator:
%\begin{equation}
%\e_{k+1} = \mathcal{T}(\bb - A\x_k + \mu^{-1}\blambda_k, \mu^{-1}).
%\end{equation}
%To solve the subproblem associated with $\x$, we
%apply a first-order $\ell^1$-minimization method,
%called \emph{fast iterative shrinkage-threshold algorithm}
%(FISTA) \cite{BeckA2009}. The main idea of FISTA is to
%iteratively minimize a quadratic approximation $Q(\x, \z)$ to
%$L_{\mu} (\x,\e_{k+1},\blambda_k)$ around a point $\z$, which is
%carefully chosen in order to achieve a good convergence
%rate. We summarize the entire ALM
%algorithm as Algorithm~\ref{alg:alm}, where $\gamma$ denotes the
%largest eigenvalue of the matrix $A^TA$. For the choice of parameter $\mu$, we take the same strategy as
%in \cite{YangJ2009-pp} and set $\mu = 2m / \|\bb\|_1$.

%\begin{algorithm}[t]
%\caption{\bf (Augmented Lagrange Multiplier Method for Global
%Recognition)}
%\begin{algorithmic}[1]
%\begin{small}
%\STATE {\bf Input:} $\bb \in \Re^m$, $A \in \Re^{m \times n}$,
%$\x_1 = \mathbf{0}$, $\e_1 = \bb$, $\blambda_1 =
%\mathbf{0}$.
%\WHILE{not converged ($k = 1,2,\ldots$)}
%\STATE $\e_{k+1} = \mathcal{T}\left(\bb - A\x_k +
%\frac{1}{\mu}\blambda_k, \frac{1}{\mu}\right)$;
%\STATE $t_1\leftarrow 1$, $\z_1 \leftarrow \x_k$, $\w_1 \leftarrow \x_k$;
%\WHILE{not converged ($l = 1,2,\ldots$)}
%\STATE $\w_{l+1} \leftarrow \mathcal{T}\left(\z_l +
%\frac{1}{\gamma}A^T\left(\bb - A\z_l - \e_{k+1} +
%\frac{1}{\mu}\blambda_k\right), \frac{1}{\mu\gamma}\right)$;
%\STATE $t_{l+1} \leftarrow \frac{1}{2}\left( 1 +
%\sqrt{1+4t_l^2}\right)$;
%\STATE $\z_{l+1} \leftarrow \w_{l+1} + \frac{t_l - 1}{t_{l+1}}(\w_{l+1} - \w_l)$;
%\ENDWHILE
%\STATE $\x_{k+1} \leftarrow \w_{l}$,  \; $\blambda_{k+1} \leftarrow \blambda_k + \mu (\bb - A\x_{k+1} - \e_{k+1})$;
%\ENDWHILE \STATE
%{\bf Output:} $\x^* \leftarrow \x_k, \e^* \leftarrow \e_k$.
%\end{small}
%\end{algorithmic}
%\label{alg:alm}
%%\vspace{-4mm}
%\end{algorithm}

%\subsection{ALM derivation for the Alignment Stage}
%This section provides a mathematical derivation for an Augmented Lagrange Method solver
%for the optimization problem:
%\begin{equation}
%\min_{\x, \e} \|\e\|_1 \quad \subj \quad \bb =
%A \x + \e.
%%\label{eqn:l1min_denoise}
%%\tag{\ref{eqn:l1min_denoise}}
%\end{equation}
%For this problem, the corresponding augmented Lagrangian function is:
%\begin{equation}
%L_\mu (\x,\e,\blambda) = \|\e\|_1 + \langle \blambda, \bb-A\x - \e \rangle + \frac{\mu}{2} \|\bb - A\x - \e\|_2^2,
%\end{equation}

%For the alternating minimizations, while the subproblem associated with $\e$
%has the same closed-form solution as before, the subproblem associated with
%$\x$ can now be solved in closed form:
%\begin{eqnarray}
%\x^* = \argmin_{\x} \langle \blambda, \bb-A\x - \e \rangle + \frac{\mu}{2} \|\bb - A\x - \e\|_2^2 \\
%\x^* = \argmin_{\x} -\blambda^T A x + \frac{\mu}{2} (\x^T(A^T A) \x + 2(\e-\bb)^T A \x + (\e-\bb)^T (\e-\bb)) \\
%\x^* = \argmin_{\x} -\blambda^T A x + \frac{\mu}{2} (\x^T(A^T A) \x + 2(\e-\bb)^T A \x) \\
%\x^* = \argmin_{\x} (-\blambda^T -\bb -\frac{\blambda}{\mu}^T A x + \frac{1}{2} \x^T (A^T A) \x \\
%0 = A^T(\e - \bb -\frac{\blambda}{mu}) + (A^T A) \x^*\\
%(A^T A) \x^* =  A^T (\bb - \e + \frac{\blambda}{mu})\\
%\x^* =  (A^T A)^{-1} A^T (\bb - \e + \frac{\blambda}{mu})\\
%\x^* =  A^\Dagger (\bb - \e + \frac{\blambda}{mu})\\
%\end{eqnarray}
%}

%\frame{\frametitle{Complete ALM Algorithm}
%\begin{algorithmic}
%\WHILE{not converged}
%\WHILE{not converged}
%\STATE $[\e_{j+1}]_i  =  \shrink \left(\left[ \bb + \frac{1}{\mu_k}\blambda_k - A\x_j \right] _i, \frac{W_{ii}}{\mu_k}\right), \quad i = 1,2,\ldots,m$
%\STATE $t_1 \leftarrow 1$, $\z_1 \leftarrow \x_j$
%\WHILE{not converged}
%\STATE $\x_{l+1}  =  \shrink\left(\z_l - \frac{1}{\tau}A^T\left(A\z_l + \e_{j+1} - \bb - \frac{1}{\mu_k}\blambda_k\right), \frac{1}{\mu_k\tau}\right)$
%\STATE $t_{l+1} = 0.5\left( 1 + \sqrt{1+4t_l^2}\right)$
%\STATE $\z_{l+1} = \x_{l+1} + \frac{t_l - 1}{t_{l+1}}(\x_{l+1} - \x_l)$
%\ENDWHILE
%\ENDWHILE
%\STATE $\blambda_{k+1} = \blambda_k + \mu_k (\bb - A\x_{k+1} - \e_{k+1})$
%\STATE $\mu_{k+1} = \rho\cdot\mu_k$
%\ENDWHILE
%\end{algorithmic}
%}

