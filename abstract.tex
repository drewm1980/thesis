% From prelim document
Most contemporary face recognition algorithms work well under laboratory
conditions but degrade when tested in less-controlled environments. In order to
achieve useful recognition rates a recognition system needs to simultaneously
handle variations in illumination, alignment, and occlusion. This thesis
proposes a conceptually simple and practical face recognition system that
achieves a high degree of robustness and stability to all these variations.
First a flexible system for acquiring well registered training images under
many illuminations is demonstrated.  An alignment and recognition system
simultaneously searches for a linear combination of the training images and a
corresponding warping of the testing image that results in an image error that
is sparse.  To better handle severe occlusions an extension to the algorithm is
proposed that makes use of the knowledge that occluded pixels tend to be
spatially correlated.  Finally, several planned techniques for improving both
the execution speed and recognition accuracy of the algorithm are discussed.

% From PAMI
%Many classic and contemporary face recognition algorithms work well on public
%data sets, but degrade sharply when they are used in a real recognition
%system. This is mostly due to the difficulty of simultaneously handling
%variations in illumination, image misalignment, and occlusion in the test
%image. We consider a scenario where the training images are well controlled,
%and test images are only loosely controlled.  We propose a conceptually simple
%face recognition system that achieves a high degree of robustness and
%stability to illumination variation, image misalignment, and partial
%occlusion. The system uses tools from sparse representation to align a test
%face image to a set of frontal training images.  The region of attraction of
%our alignment algorithm is computed empirically for public face datasets such
%as Multi-PIE. We demonstrate how to capture a set of training images with
%enough illumination variation that they span test images taken under
%uncontrolled illumination. In order to evaluate how our algorithms work under
%practical testing conditions, we have implemented a complete face recognition
%system, including a projector-based training acquisition system. Our system
%can efficiently and effectively recognize faces under a variety of realistic
%conditions, using only frontal images under the proposed illuminations as
%training.
