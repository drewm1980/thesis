% TODO introduce speed objective earlier.  From prelim document
Most contemporary face recognition algorithms work well under laboratory
conditions but degrade when tested in less-controlled environments. In order to
achieve useful recognition rates a recognition system needs to simultaneously
handle variations in illumination, alignment, and occlusion. This thesis
proposes a conceptually simple and practical face recognition system that
achieves a high degree of robustness and stability to all these variations.
First a flexible projector-based system for acquiring well registered training
images under many illuminations is demonstrated.  An alignment and recognition
system simultaneously searches for a linear combination of the training images
and a corresponding warping of the testing image that results in an image error
that is sparse.  To better handle severe occlusions an extension to the
algorithm is proposed that makes use of the knowledge that occluded pixels tend
to be spatially correlated.  Due to the use of multiple face images as features
and as the non-smooth nature of the optimization problems, these techniques
have far greater computational requirements than techniques that extract
low-dimensional features.  Both algorithmic improvements and optimized
implementations for modern parallel computing architectures are investigated in
order to a build a system capable of perform highly accurate and robust
recognition while remaining fast enough for use in access control system.  
%The work presented here has become the reference algorithm for a new class of
%recognition systems that have been further developed by other researchers.


% From PAMI
%Many classic and contemporary face recognition algorithms work well on public
%data sets, but degrade sharply when they are used in a real recognition
%system. This is mostly due to the difficulty of simultaneously handling
%variations in illumination, image misalignment, and occlusion in the test
%image. We consider a scenario where the training images are well controlled,
%and test images are only loosely controlled.  We propose a conceptually simple
%face recognition system that achieves a high degree of robustness and
%stability to illumination variation, image misalignment, and partial
%occlusion. The system uses tools from sparse representation to align a test
%face image to a set of frontal training images.  The region of attraction of
%our alignment algorithm is computed empirically for public face datasets such
%as Multi-PIE. We demonstrate how to capture a set of training images with
%enough illumination variation that they span test images taken under
%uncontrolled illumination. In order to evaluate how our algorithms work under
%practical testing conditions, we have implemented a complete face recognition
%system, including a projector-based training acquisition system. Our system
%can efficiently and effectively recognize faces under a variety of realistic
%conditions, using only frontal images under the proposed illuminations as
%training.
%\begin{abstract} Recently a family of promising face recognition algorithms
%based on sparse representation and $\ell_1$-minimization ($\ell_1$-min) have been
%developed.  These algorithms have not yet seen commercial application, 
%largely due to higher computational cost compared to other traditional
%algorithms. This paper studies techniques for leveraging the 
%massive parallelism available in GPU and CPU hardware to accelerate
%$\ell_1$-min based on augmented Lagrangian method (ALM) solvers. 
%For very large problems, the GPU is faster due to higher memory bandwidth, while 
%for problems that fit in the larger CPU L3 cache, the CPU is faster.
%On both architectures, the proposed implementations significantly outperform
%naive library-based implementations, as well as previous systems. 
%The source code of the
%algorithms will be made available for peer evaluation.  \end{abstract}

