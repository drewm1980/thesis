Many classic and contemporary face recognition algorithms work well on public
data sets, but degrade sharply when they are used in a real recognition system.
A major cause of this is the difficulty of simultaneously handling variations
in illumination, image misalignment, and occlusion in the test image; while in
some applications the gallery images can be well controlled, the test images
are only loosely controlled.  This thesis describes a conceptually simple but
computationally intense face recognition system that achieves a high degree of
robustness and stability to illumination variation, image misalignment, and
partial occlusion, along with optimized parallel implementations.

First, well registered training images taken under many illumination directions
are captured using a novel projector-based acquisition system.  The recognition
system then uses tools from sparse representation to align a test face image to
a set of frontal training images.  To better handle severe occlusions, an
extension to the algorithm is described that makes use of the knowledge that
occluded pixels tend to be spatially correlated.  Due to the use of multiple
face images as features and as the non-smooth nature of the optimization
problems, these techniques have far greater computational requirements than
techniques that extract low-dimensional features.  Several custom $\ell_1$
solvers are presented that achieve faster convergence on face data than general
solvers.  Optimized implementations for modern parallel computing architectures
are investigated in order to build a system capable of performing highly
accurate and robust recognition while remaining fast enough for use in access
control systems.  Optimized parallel implementations for contemporary CPU and
GPU hardware are demonstrated to achieve near real-time face recognition for
access control applications with hundreds of gallery users.



