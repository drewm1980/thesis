\section{Introduction}
%
% TODO dig up old discussions of illumination wrt basri.
% TODO dig up discussion of coefficient positivity

As introduced in Chapter \ref{chap:introduction}, SRC \cite{Wright2009-PAMI} 
achieves impressive recognition results on aligned images,
it does not deal with misalignment between the test and training
images, and it requires a rich set of illuminations in the gallery images for
good performance.  The need for proper handling of image alignment and
illumination {\em simultaneously} is illustrated by an example in Figure \ref{fig:promo}.  
The task is to
identify the girl among 20 subjects. If the test face image, obtained from
an off-the-shelf face detector, has even a small amount of registration error
against the training images (caused by mild pose, scale, or misalignment), the
sparse representation obtained using the method of \cite{Wright2009-PAMI} is no
longer informative, even if sufficient illuminations are present in the
training, as shown in Figure \ref{fig:promo}(top). Additionally, in order to
span the illuminations of a typical indoor (or outdoor) environment,
illuminations from behind the subject are needed in the training set.
Otherwise, even for perfectly aligned test images, the sparse representation
obtained using \cite{Wright2009-PAMI} will not necessarily be sparse or
informative, as shown by the example in Figure \ref{fig:promo}(middle).
Clearly, both good alignment, as well as sufficient training images are needed
to ensure success of the sparsity-based recognition method proposed by
\cite{Wright2009-PAMI}.  This chapter demonstrates a system for handling alignment
and illumination simultaneously in the sparse representation framework,
bringing the method proposed in \cite{Wright2009-PAMI} closer to practical use.

\newcommand{\tempheight}[0]{1.1in}
\begin{figure}
\centering \begin{tabular}{cc}
\includegraphics[height=\tempheight]{figures_pami/promo/case1/detector.png}&
\hspace{3mm}
\includegraphics[height=\tempheight]{figures_pami/promo/case1/sci_with_axis_face_case1.png}
\\ \includegraphics[height=\tempheight]{figures_pami/promo/alignment_and_detector.png}&
\hspace{3mm}
\includegraphics[height=\tempheight]{figures_pami/promo/case2/sci_with_axis_face_case2.png}
\\ \includegraphics[height=\tempheight]{figures_pami/promo/case3/alignment.png} &
\hspace{3mm}
\includegraphics[height=\tempheight]{figures_pami/promo/case3/sci_with_axis_face_case3.png}
\end{tabular} \caption{\small{\bf Effects of registration and illumination on
Recognition}. In this example we identify the girl among 20 subjects, by
computing the sparse representation of her input face with respect to the
entire training set. The absolute sum of the coefficients associated with each
subject is plotted on the right. The weighted sum of the 
subject's training images using the associated sparse coefficients is also shown.
The red line (cross) corresponds to her true identity, subject 12. {\bf Top:} The
input face is from Viola and Jones' face detector (the black box) and all 38
illuminations specified in Section \ref{sec:illumination} are used in the
training.  {\bf Middle:} The input face is well-aligned (the white box) with
the training by our algorithm specified in Section \ref{sec:registration} but
only 24 frontal illuminations are used in the training for recognition (see
Section \ref{sec:illumination}). {\bf Bottom:} The input face is well aligned and
a sufficient set (all 38) of
illuminations are used in the training. Both are necessary for correct recognition
using SRC.}\label{fig:promo}
\end{figure}

\subsection{Relation to Earlier Work on Image Registration}

In holistic recognition algorithms (algorithms that use images themselves as
features) correspondence between points in the test image and in the gallery
images must be achieved.  A long line of research exists on using Active
Appearance Models \cite{Cootes2001-PAMI}, and the closely related Active Shape
Models \cite{cootes1992active} to register images against a relatively
high-dimensional model of plausible face appearances, often leveraging
face-specific contours.  While these model-based techniques have advantages in
dealing with variations in expression and pose, they may add unnecessary
complexity to applications where subjects normally present a neutral face or
only have moderate expression. Instead, this work focuses on classes of
deformations with far fewer degrees of freedom, i.e. similarity transformations
or perspective transformations.  Iterative registration in this spirit of this
work dates at least back to the Lucas-Kanade algorithm
\cite{lucas1981iterative}.

Whereas much of the early work on image registration is aimed at the problem of
registering nearly identical images, say by minimizing a sum of squared
distances or maximizing normalized correlation, face recognition applications
must confront several physical factors simultaneously: misalignment,
illumination variations, and corrupted pixels.  As will be discussed further
below, illumination variation can be dealt with by expressing the test image as
a linear combination of an appropriate set of training images. Similar
representations have been exploited in illumination-robust tracking (e.g.,
\cite{Belhumeur1999-PAA,Murase1995-IJCV}).  For robustness to gross errors, the
$\ell^1$-norm of the residual is a more appropriate objective function than the
classical $\ell^2$-norm. Its use here is loosely motivated by theoretical
results due to Cand\`{e}s and Tao \cite{CandesE2005-IT} (see also
\cite{Wright2008-IT}). These two observations motivate the reformulation of the
registration problem as the search for a set of transformations and
illumination coefficients that minimize the $\ell^1$-norm of the representation
error.  The proposed alignment system uses a generalized Gauss-Newton method
which solves a sequence of affine-constrained $\ell^1$-norm minimization
problems \cite{Osborne1990-JAMSSB,Jittorntrum1980-NM}. Each of these problems
can also be solved efficiently using recently developed first-order techniques
for $\ell^1$-minimization, which are reviewed in \cite{YangA2010-pp}.

% Illumination
Researchers have tried various techniques to deal with illumination variation.
In almost all recognition algorithms where only a single gallery image is
available per individual, illumination effects are regarded as a nuisance that
must be removed before the algorithm can continue.  This is typically done by
making statistical assumptions about how illumination affects the image, and
using those assumptions to extract a new representation that is claimed to be
illumination invariant.  Recent examples include \cite{chen2006total} and
\cite{zhou2007appearance}.  Despite these efforts, truly
illumination-invariant features remain difficult to obtain from a single input
image.  Clearly, if one has the luxury of designing the acquisition system
and the application demands a high recognition rate,
it is then unwise to limit the gallery to a
single image per person.  The proposed recognition system therefore takes the strategy of sampling many
gallery images of each individual under varying illuminations.  These images
are used as the basis for either a convex cone model
\cite{Georghiades2001-PAMI,belhumeur1998set}, or a subspace model
\cite{Basri2003-PAMI}.  Images are captured using a simple-to-construct
projector based light stage.  While similar systems have been used for
other applications, to our knowledge, this
system is the first to use projectors to indirectly illuminate a subject's face for
the purpose of face recognition.

\subsection{Contributions} This chapter demonstrates how registration and
illumination can be simultaneously addressed within a robust sparse
representation framework. Face registration, despite being a challenging
nonlinear problem, can be solved by a series of linear programs that
iteratively minimize the sparsity of the registration error. This leads to an
efficient and effective alignment algorithm for face images that works for a
large range of variation in translation, rotation, and scale, even when the
face is only partially visible due to eyeglasses, closed eyes and open mouth,
sensor saturation, etc.  A sufficient set of training illuminations that is
capable of linearly representing typical indoor and outdoor lighting is
determined empirically, and a practical hardware system based on synchronized
cameras and projectors is developed for capturing them.

%A key part of the system is exploiting
%an important property of the imaging process:  there is a linear mapping
%between the space of illuminations of an object, and the space of images of
%that object taken under the same pose.  This makes it possible to effectively
%model the testing image as a linear superposition of a large (and well chosen)
%set of training images.  This idea is certainly not new; indeed it has been in
%use for face recognition for roughly two decades, \cite{Turk1991-CVPR}.
%However, traditional algorithms that rely on this property of the image
%formation process have tended to perform very badly in the face of occlusions
%and when highly quality training images are unavailable.  

The chapter then demonstrates the effectiveness of the proposed new
methods with a complete face recognition system that is {\em
simple, stable, and scalable}. The proposed system performs
robust automatic recognition of subjects from loosely
controlled probe images taken both indoors and outdoors,
using a gallery of
frontal views of the subjects' faces under the proposed
illuminations. An off-the-shelf face
detector\footnote{We use the OpenCV
implementation of the Viola and Jones' face detector
\cite{Viola2004-IJCV}.} is used to detect faces in the test images.

Extensive experiments are conducted on the proposed system with
both public databases and a face database that is collected by
the proposed acquisition system. The experimental results on
large-scale public face databases show that the algorithm
indeed achieves very good performance on these databases,
exceeding or competing with the state-of-the-art algorithms.
Additionally, the experimental results on the private database
clearly demonstrate that the recognition system not only works well with
images taken under controlled laboratory conditions, but is
capable of handling practical indoor and outdoor illuminations as well.

\noindent{\em Organization of this chapter:} Section \ref{sec:registration},
presents a derivation of the robust registration and recognition algorithm within the sparse
representation framework. It further elaborates on algorithmic implementation issues,
conducts region of attraction experiments with respect to both 2D in-plane
deformation and 3D pose variation, and discusses its relationship to existing
work. Section \ref{sec:illumination} is dedicated to the training acquisition
system. This system is used to investigate empirically how many training
illuminations are required to handle practical illumination variations, and to
suggest a sufficient set of 38 training illuminations. Extensive experiments on
a large-scale public database and on a newly gathered database are conducted in Section
\ref{sec:multipie} and Section \ref{sec:own-data}, respectively, to verify the
proposed system. 

\section{Robust Alignment}\label{sec:registration} As demonstrated in Figure
\ref{fig:promo}(top), the main limitation of the {\em Sparse Representation and
Classification} (SRC) algorithm of \cite{Wright2009-PAMI} is the assumption of
pixel-accurate alignment between the test image and the training set. This
leads to brittleness under pose and misalignment, making it inappropriate for
deployment outside a laboratory setting. The goal of this section is to show
how this weakness can be rectified while still preserving the conceptual
simplicity and good recognition performance of SRC.

SRC assumes access to a database of multiple registered
training images per subject, taken under varying illuminations.
The images of subject $i$, stacked as vectors, form a matrix
$A_i \in \Re^{m \times n_i}$. Taken together, all of the images
form a large matrix $A = [ A_1 \mid A_2 \mid \dots \mid A_K ]
\in \Re^{m \times n}$. As argued in \cite{Wright2009-PAMI}, a
well-aligned test image $\y_0$ can be represented as a sparse
linear combination $A \x_0$ of all of the images in the
database,\footnote{This assumes that the training illuminations are sufficient. The next section will address how to ensure illumination
sufficiency.} plus a sparse error $\e_0$
due to corrupted pixels. The sparse representation can be recovered by
minimizing the $\ell^1$-norm\footnote{The $\ell^1$-norm of a
vector, denoted by $\|\cdot\|_1$, is the sum of absolute values of its entries.} of
$\x$ and $\e$:
\begin{equation}
\min_{\x,\e} \, \| \x \|_1 + \| \e\|_1 \quad \subj \quad \y_0 = A \x + \e.
\label{eqn:robust-l1}
\end{equation}
Now suppose that $\y_0$ is subject to some pose or
misalignment, so that instead of recording $\y_0$, the camera captures
a warped image $\y = \y_0 \circ \tau^{-1}$, for some
transformation $\tau \in T$ where $T$ is a finite-dimensional
group of transformations acting on the image domain.  The
transformed image $\y$ no longer has a sparse representation of
the form $\y = A \x_0 + \e_0$, and naively applying the
algorithm of \cite{Wright2009-PAMI} is no longer appropriate,
as seen in Figure \ref{fig:promo}(top).

\subsection{Batch and Individual Alignment} If the
true deformation $\tau^{-1}$ can be found, then
its inverse $\tau$ can be applied to the test image and it again becomes
possible to find a sparse representation of the resulting
image, as $\y \circ \tau = A \x_0 + \e_0$.\footnote{In the terminology of \cite{baker2004lucas}, this formulation is ``Forward Additive''.}
  This sparsity
provides a strong cue for finding the correct deformation
$\tau$: conceptually, one would like to seek a transformation
$\tau$ that allows the sparsest representation, via
\begin{equation} \label{eqn:L1-L1-conceptual}
\hat{\tau} = \arg\hspace{-2.5mm}\min_{\x,\e,\tau \in T} \| \x \|_1 + \| \e \|_1 \quad \subj \quad \y \circ \tau = A \x + \e.
\end{equation}
For fixed $\tau$, this problem is jointly convex in $\x$ and
$\e$. However, as a simultaneous optimization over the
coefficients $\x$, error representation $\e$, and
transformation $\tau$, it is a difficult, non-convex
optimization problem. One source of difficulty is the presence
of multiple faces in the matrix $A$:
\eqref{eqn:L1-L1-conceptual} has many local minima that
correspond to aligning $\y$ to different subjects. In this
sense, the misaligned recognition problem differs from the
well-aligned version studied in \cite{Wright2009-PAMI}. For the
well-aligned case, it is possible to directly solve for a
global representation, with no concern for local minima. With
possible misalignment, it is more appropriate to seek the best
alignment of the test face with each subject $i$:
\begin{equation} \label{eqn:per-subject-L1}
\hat \tau_i = \arg\hspace{-2.5mm}\min_{\x,\e,\tau_i \in T} \| \e \|_1 \quad \subj \quad \y \circ \tau_i = A_i \x + \e.
\end{equation}
It no makes sense to penalize $\| \x \|_1$, since $A_i$ consists of
only images of subject $i$ and therefore $\x$ is no longer expected to
be sparse.

\subsection{Alignment via Sequential $\ell^1$-Minimization} While the problem
\eqref{eqn:per-subject-L1} is still non-convex, for cases of practical interest
in face recognition, a good initial guess for the transformation is available,
e.g., from the output of a face detector. This initialization can be refined to
approach an estimate of the true transformation by repeatedly linearizing about  the
current estimate of $\tau$, and seeking representations of the form:
\begin{equation}
\y\circ \tau + J \Delta \tau = A_i \x + \e.
\end{equation}
Here, $J = \frac{\partial}{\partial \tau} \y \circ \tau$ is the Jacobian of $\y
\circ \tau$ with respect to the transformation parameters $\tau$, and $\Delta
\tau$ is the step in $\tau$. The above equation is under-determined if the
registration error $\e$ is allowed to be arbitrary. At the correct alignment it
is expected that the test image only differs from $A_i \x$ only for the
minority of the pixels corrupted by occlusions. Thus, the algorithm computes a
deformation step $\Delta \tau$ that best sparsifies the registration error
$\e$, in terms of its $\ell^1$-norm:
\begin{equation}
\Delta\hat{\tau}_1 = \arg\hspace{-3.5mm}\min_{\x,\e,\Delta\tau \in T} \| \e \|_1 \quad \subj \quad \y\circ\tau + J \Delta \tau = A_i \x + \e.
\label{eqn:L1-align}
\end{equation}
This is different from the popular choice that
minimizes the $\ell^2$-norm of the registration error:
\begin{equation}
\Delta\hat{\tau}_2 = \arg\hspace{-3.5mm}\min_{\x,\e,\Delta\tau \in T} \| \e \|_2 \quad \subj \quad \y\circ\tau + J \Delta \tau = A_i \x + \e,
\label{eqn:L2-align}
\end{equation}
which is also equivalent to finding the deformation step
$\Delta  \tau$ by solving the least-square problem:
$\min_{\x,\Delta \tau} \|\y \circ \tau + J\Delta \tau - A_i \x
\|_2$. Empirically, if there is only small noise
between $\y_0$ and $A_i\x$, both \eqref{eqn:L1-align} and
$\eqref{eqn:L2-align}$ have similar performance.  However, if
there are occlusions in $\y_0$, sequential
$\ell^1$-minimization \eqref{eqn:L1-align} is significantly
better than sequential $\ell^2$-minimization
\eqref{eqn:L2-align}. Figure \ref{fig:L1-L2-align} shows an
example.

The scheme \eqref{eqn:L1-align} can be viewed as a generalized Gauss-Newton
method for minimizing the composition of a non-smooth objective function (the
$\ell^1$-norm) with a differentiable mapping from transformation parameters to
transformed images. Such algorithms date at least back to the 1970's
\cite{Cromme1978-NM,Jittorntrum1980-NM}, and continue to attract attention
today \cite{Lewis2008-TR}. While a detailed discussion of their properties is
outside the scope of this thesis, it is worth mentioning that the scheme
\eqref{eqn:L1-align} is known to converge quadratically in the neighborhood of
any local optimum of the $\ell^1$-norm. In practice, this means that $\approx$
10 to 15 iterations suffice to reach the desired solution. The
interested reader is referred to \cite{Jittorntrum1980-NM,Osborne1990-JAMSSB} and the
references therein.

\renewcommand{\tempheight}[0]{1.0in}
\begin{figure}
\centering
{
\begin{tabular}{cccc}
\includegraphics[height=\tempheight]{figures_pami/L1_cropped} &
\includegraphics[height=\tempheight]{figures_pami/y_warp_L1} &
\includegraphics[height=\tempheight]{figures_pami/y_hat_L1} &
\includegraphics[height=\tempheight]{figures_pami/e_L1} \\
\includegraphics[height=\tempheight]{figures_pami/L2_cropped} &
\includegraphics[height=\tempheight]{figures_pami/y_warp_L2} &
\includegraphics[height=\tempheight]{figures_pami/y_hat_L2} &
\includegraphics[height=\tempheight]{figures_pami/e_L2} \\
(a) & (b) & (c) & (d)
\end{tabular}}
\caption{\small{\bf Comparing alignment of a subject wearing sunglasses by
$\ell^1$ and $\ell^2$ minimization.}
{\bf Top:} alignment result of minimizing $\|\e\|_1$; {\bf Bottom:}
result of minimizing $\|\e\|_2$. (a) {\em Green (dotted):} Initial face boundary
given by the face detector, {\em Red (solid):} Alignment result shown on the same
face; (b) warped testing image using the estimated transformation $\y_0$;
(c) reconstructed face $A_i\x$ using the training; (d) image of error $\e$. }\label{fig:L1-L2-align}
\end{figure}

In addition to normalizing the training images (which is done
once), it is important to normalize the warped testing image
$\y \circ \tau$ as the algorithm runs.  Without normalization,
the algorithm may fall into a degenerate global minimum
corresponding to zooming in on a dark region of the test
image.  Normalization is done by replacing the linearization of
$\y \circ \tau$ with a linearization of the normalized version
$\tilde \y(\tau) = \frac{\y \circ \tau}{\|\y \circ \tau\|_2}$.
The proposed alignment algorithm can be easily extended to work
in a {\em multiscale} fashion, with benefits both in
convergence behavior and computational cost.  The alignment
algorithm is simply run to completion on progressively less
downsampled versions of the training and testing images, using
the result of one level to initialize the next.

\subsection{Robust Recognition by Sparse Representation} Once
the best transformation $\tau_i$ has been computed for each
subject $i$, the training sets $A_i$ can be aligned to $\y$,
and a global sparse representation problem of the form
\eqref{eqn:robust-l1} can be solved to obtain a discriminative
representation in terms of the entire training set. Moreover,
the per-subject alignment residuals $\| \e \|_1$ can be used to
prune unpromising candidates from the global optimization,
leaving a much smaller and more efficiently solvable problem.
The complete optimization procedure is summarized as Algorithm
\ref{alg:deformable-src}. The parameter $S$ is the number of subjects
considered together to provide a sparse representation for the
test image. If $S = 1$, the algorithm reduces to classification
by registration error; but considering the test image might be
an invalid subject, $S=10$ is typically chosen. Since valid
images have a sparse representation in terms of this larger
set, invalid test images can be rejected using the {\em sparsity
concentration index} proposed in \cite{Wright2009-PAMI}.
The function $\delta_i(\x)$ in Algorithm \ref{alg:deformable-src}
selects coefficients from the vector $\x$ corresponding to subject $i$.

% WORKING HERE
Another important free parameter in Algorithm
\ref{alg:deformable-src} is the class of deformations $T$. In
the experiments presented here, 2D similarity (i.e. 2D rigid transformations)
transformations are used, $T = \mathbb{SE}(2)\times \Re_+$\footnote{Here, SE stands for Special
Euclidean.  The $\Re_+$ accounts for the scale.}, for removing alignment error incurred by face
detector, or 2D projective transformations, $T =
\mathbb{GL}(3)\footnote{Here, GL stands for General Linear.
This class of transformations is able to represent distortion
in a perspective image of a planar object.}$, for handling some
pose variation.

In Algorithm \ref{alg:deformable-src}, we also implement a simple heuristic which improves the
performance of our system, based on the observation that the face detector output may be poorly centered on
the face, and may contain a significant amount of the background.  Therefore, before the recognition stage,
instead of aligning the training sets to the
original $\y$ directly obtained from the face detector, we
compute an average transformation $\bar{\tau}$ from $\tau_{k_1},
\tau_{k_2}, \ldots, \tau_{k_S}$ of the top $S$ classes, which
is believed to be better centered, and update $\y$ according to
$\bar{\tau}$. For the 2D similarity transformations, which are
used in our system when initialized by the face detector, a
transformation $\tau$ can be parameterized as $\tau = (\tau^1,
\tau^2, \tau^3, \tau^4)$, where $\tau^1$ and $\tau^2$ represent
the translations in $x$- and $y$-axis, $\tau^3$ represents the
rotation angle and $\tau^4$ represents the scale. Then the
average transformation is simply obtained by taking the
component-wise mean:
\begin{displaymath}
\bar{\tau}^i = (\tau_{k_1}^i + \tau_{k_2}^i + \cdots +
\tau_{k_S}^i) / S, i = 1,2,3,4.
\end{displaymath}
Finally, the training sets are aligned to the new $\y$.

\begin{algorithm}[thb]
\caption{\bf (Deformable Sparse Recovery and Classification for
Face Recognition)} \label{alg:deformable-src}
\begin{algorithmic}[1]
\begin{small}
\STATE {\bf Input:} Training images $\{A_i \in \Re^{m\times n_i}\}_{i=1}^K$ for $K$ subjects,  a test image
$\y\in\Re^m$ and a deformation group $T$.
\STATE
{\bf for} each subject $i$,
\STATE \hspace{3mm} $\tau^{(0)}
\leftarrow I$.
\STATE \hspace{3mm} {\bf while} not converged $(j=1,2,\ldots)$ {\bf do}
\STATE \hspace{6mm}
$\tilde \y(\tau) \leftarrow \frac{\y \circ \tau}{\|\y \circ
\tau\|_2}$; \;\;\; $J \leftarrow  \frac{\partial}{\partial
\tau} \tilde\y(\tau)  \bigr|_{\tau^{(j)}} $;
%\STATE \hspace{6mm} $(\hat \x, \hat \e, \Delta \tau) \leftarrow \left\{\begin{array}{l} \arg \min_{\x,\e,\Delta \tau} \| \e \|_1 \\  \subj \; \y + J \Delta \tau = A_k \x + \e \end{array}\right.$
\STATE \hspace{6mm} $ \Delta \tau =  \arg\min \; \| \e \|_1  \;
\subj \; \tilde \y + J \Delta \tau = A_i \x + \e.$
\STATE
\hspace{6mm} $\tau^{(j+1)} \leftarrow \tau^{(j)} + \Delta
\tau$;
\STATE \hspace{3mm} {\bf end while} \STATE {\bf end} \STATE Keep
the top $S$ candidates $k_1, \ldots, k_S$ with the smallest
residuals $\|\e\|_1$. \STATE Compute an average transformation
$\bar{\tau}$ from $\tau_{k_1}, \tau_{k_2}, \ldots, \tau_{k_S}$.
\STATE Update $\y \leftarrow \y \circ \bar{\tau}$ and $\tau_i
\leftarrow \tau_i \cdot \bar{\tau}^{-1}$ for $i = k_1, \dots,
k_S$. \STATE Set $A \leftarrow \big[ A_{k_1} \circ
\tau_{k_1}^{-1} \mid A_{k_2} \circ \tau_{k_2}^{-1} \mid \dots
\mid A_{k_S} \circ \tau_{k_S}^{-1} \big]$. \STATE Solve the
$\ell^1$-minimization problem: \hspace{2mm}
$\hat{\x} = \arg \min_{\x, \e} \| \x \|_1 + \|\e\|_1 \;\; \subj \;\; \y = A \x + \e.$
\STATE Compute residuals $r_i(\y) = \| {\y} - {A}_i \, \delta_i(\hat{\x}) \|_2$ for $i = k_1, \dots, k_S$.
\STATE {\bf Output:} $\mbox{identity}(\y) = \arg\min_i r_i(\y)$.
\end{small}
\end{algorithmic}
\end{algorithm}
%\vspace{-4mm}


The transformation $\tau$ defines a mapping between the
coordinates of pixels in the large original image and a smaller
(un)warped image. The pixels of the small image are stacked into a vector. To prevent aliasing
artifacts in the downsampled image, one should apply a
smoothing filter to the original image. For a simple implementation, a rectangular window with
regular sampling can used, but in general, the small image need
not be regularly sampled in pixel coordinates.  For example,
the sample locations could be arbitrarily selected from within a
``face shaped'' area. We will discuss how choosing
different windows can affect the performance of our algorithm
in Section \ref{sec:multipie}.

\subsection{System Implementation}
The runtime of Algorithm~\ref{alg:deformable-src} is dominated
by the time spent solving two qualitatively similar $\ell_1$ minimization problems.
We have developed custom solvers for this purpose based on
\emph{Augmented Lagrange Multiplier} (ALM) algorithm.
We have selected this algorithm because it strikes the best
balance between speed, accuracy, and scalability for our problem out of
many algorithms that we have tested. We refer the reader to our
supplementary materials for a more in-depth discussion of our
solvers.
For a more detailed discussion of competing
approaches, we refer the interested reader to
\cite{YangA2010-pp}.
On a Mac Pro with
Dual-Core 2.66GHz Xeon processors and 4GB memory,
running on our database containing images size $80\times 60$
pixels from 109 subjects under 38 illuminations,
our C implementation of Algorithm~\ref{alg:deformable-src} takes
about 0.60 seconds per subject for alignment and about 2.0
seconds for global recognition. Compared to the highly
customized interior point method first presented in 
\cite{Wagner2009-CVPR}, this new algorithm is
only slightly faster for per subject alignment. However, it is
much simpler to implement and it achieves a
\emph{speedup of more than a factor of 10} for global
recognition!

\subsection{Experiments on Region of Attraction} We will now present three
experimental results demonstrating the effectiveness of the individual
alignment procedure outlined in the previous section. They show the sufficiency
of the region of attraction, verify effectiveness of the multiscale extension,
and show stability to small pose variations.  We delay large-scale recognition
experiments to Sections \ref{sec:multipie} and \ref{sec:own-data}, after we
have discussed the issue of illumination in the next section.

\noindent {1) \em 2D Deformation.}  We first verify the
    effectiveness of our alignment algorithm with images
    from the CMU Multi-PIE Database \cite{Gross2008-FGR}.
    We select all the subjects in Session 1, use 7
    illuminations per person from Session 1 for training,
    and test on one new illumination from Session
    2.\footnote{The training are illuminations $\{0, 1, 7,
    13, 14, 16, 18\}$ of \cite{Gross2008-FGR}, and the
    testing is the illumination 10. } We manually select
    eye corners in both training and testing as the ground
    truth for registration. We downsample the images to
    $80\times 60$ pixels\footnote{Unless otherwise stated,
    this will be the default resolution at which we prepare
    all our training and testing datasets.} and the distance between the two outer
    eye corners is normalized to be 50 pixels for each
    person. We introduce artificial deformation to the
    testing image with a combination of translation,
    rotation and scaling. We further use the alignment
    error $\|\e\|_1$ as an indicator of success. Let $r_0$
    be the alignment error obtained by aligning a test
    image to the training images without any artificial
    perturbation. When the test image is artificially
    perturbed and aligned, resulting in an alignment error
    $r$, we consider the alignment successful if $|r - r_0 | \leq
    0.01r_0$. Figure \ref{fig:attraction} shows the
    percentage of successful registrations for all subjects
    for each artificial deformation. The results suggest
    that our algorithm works extremely well with
    translation up to 20\% of the eye distance (or 10
    pixels) in all directions and up to $30^\circ$ in-plane
    rotation. We have also tested our alignment algorithm
    with scale variation and it can handle up to 15\%
    change in scale.

We have gathered the statistics of the Viola and Jones'
 face detector on the Multi-PIE dataset. For 4,600 frontal
 images of 230 subjects under 20 different illuminations,
 using manual registration as the ground truth, the average
 misalignment error of the detected faces is about 6 pixels
 and the average variation in scale is 8\%. This falls
 safely inside the region of attraction for our alignment
 algorithm.
\newcommand{\tempheighta}[0]{1.1in}
\newcommand{\tempheightb}[0]{0.9in}
\begin{figure*}
\centering
{
\begin{tabular}{ccc}
\includegraphics[height=\tempheighta]{figures_pami/x_y_roa.png} &
\includegraphics[height=\tempheighta]{figures_pami/y_theta_roa.png} &
\includegraphics[height=\tempheighta]{figures_pami/y_s_roa.png}\\
(a)&(b)&(c)
\end{tabular}
\begin{tabular}{cccc}
\includegraphics[height=\tempheightb]{figures_pami/x_tr.png} &
\includegraphics[height=\tempheightb]{figures_pami/y_tr.png} &
\includegraphics[height=\tempheightb]{figures_pami/theta.png} &
\includegraphics[height=\tempheightb]{figures_pami/scale.png}\\
(d)&(e)&(f)& (g)
\end{tabular}
} \caption{\small{\bf Region of attraction.} Fraction of
subjects for which the algorithm successfully aligns a
synthetically perturbed test image.  The amount of translation
is expressed as a fraction of the distance between the outer
eye corners, and the amount of in-plane rotation in degrees.
{\bf Top row:} (a) Simultaneous translation in $x$ and $y$
directions. (b) Simultaneous translation in $y$ direction and
in-plane rotation. (c) Simultaneous translation in $y$
direction and scale variation. {\bf Bottom row:} (d)
Translation in $x$ direction only. (e) Translation in $y$
direction only. (f) In-plane rotation only. (g) Scale variation
only.} \label{fig:attraction} 
\end{figure*}

\noindent{2) \em Multiscale Implementation.}
Performing alignment in a multiscale fashion has two benefits: first, it provides a larger region of attraction, and second, it reduces overall computational cost. Here, we further investigate the convergence behavior of the algorithm as a function of the standard deviation $\sigma$ of the Gaussian smoothing filter and the number of scales considered.
We use the same 7 illuminations in
Session 1 as training, and all 20 illuminations in the same
session as testing. We introduce artificial deformation in
both $x$ and $y$ directions up to 16 pixels in the
$80\times 60$ frame, with a step size of 4 pixels, i.e.,
$(\Delta x, \Delta y) \in \{-16,-12,\ldots,12,16\} \times
\{-16,-12,\ldots,12,16\}$. We consider an alignment
successful if the estimated coordinates of the eye-corners
are within 1 pixel from the ground truth in the original
image.  In Figure \ref{fig:multiscale}, we report the
alignment success rate, averaged over the artificially
perturbed initial deformations, as a function of the
standard deviation of the Gaussian kernel $\sigma$, for
three choices of the number of scales. As one can see,
using multiscale indeed improves the performance, and when
3 scales are used, a smaller convolution kernel can achieve
a similar performance compared to a much larger kernel when
only 2 scales are used.
\begin{figure}
\centering
\includegraphics[width=4in]{figures_pami/multiscale.png}
\caption{\small{\bf Multiscale alignment.} This figure shows the average success rate of alignment over all possible perturbations. A smaller blur kernel can be applied to achieve certain level of performance when more scales are used.}
\label{fig:multiscale}
\end{figure}

\noindent{3) \em 3D Pose Variation.} As densely sampled pose
and
    illumination face images are not available in any of
    the public databases, including Multi-PIE, we have
    collected our own dataset using our own system (to be
    introduced in the next section). We use frontal face
    images of a subject under the 38 illuminations proposed
    in the next section as training. For testing, we
    collect images of the subject under a typical indoor
    lighting condition at pose ranging from $-90^\circ$ to
    $+90^\circ$ with step size 5.625$^\circ$, a total of 33
    poses. We use Viola and Jones' face detector to
    initialize our alignment algorithm.
Figure \ref{fig:pose-alignment} shows that our algorithm works reasonably well with poses up
to $\pm 45^\circ$.
Note that this level of out-of-plane
 pose variation is beyond what we intend to handle with our formulation.
\begin{figure}
\centering
{
\begin{tabular}{ccccc}
\includegraphics[height=1in]{figures_pami/5} &
\includegraphics[height=1in]{figures_pami/7} &
\includegraphics[height=1in]{figures_pami/09} &
\includegraphics[height=1in]{figures_pami/11} &
\includegraphics[height=1in]{figures_pami/13} \\
(a) & (b) & (c) & (d) & (e)\\
\includegraphics[height=1in]{figures_pami/15} &
\includegraphics[height=1in]{figures_pami/17} &
\includegraphics[height=1in]{figures_pami/19} &
\includegraphics[height=1in]{figures_pami/21} &
\includegraphics[height=1in]{figures_pami/3} \\
(f) & (g) & (h) & (i) & (j)
\end{tabular}
}
\caption{\small{\bf 2D Alignment of test images with different poses to frontal training images.} {\bf (a) to (i):}  plausible alignment for pose from $-45^{\circ}$
to $+45^{\circ}$. {\bf (j):} a case when the algorithm fails for an extreme pose ($>45^{\circ}$).
}\label{fig:pose-alignment} 
\end{figure}
\subsection{Comparison with Related Work}
Our modification to SRC roots solidly in the tradition of
adding deformation-robustness to face recognition algorithms
\cite{Cootes2001-PAMI,Gross2006-PAMI,Wiskott1997-PAMI}.
However, the only previous work to investigate face alignment
in the context of sparse signal representation and SRC is the
work of \cite{Huang2008-CVPR}. They consider the case where the
training images themselves are misaligned and allow one
deformation per training image. They linearize the training
rather than the test, which is computationally more costly as
it effectively triples the size of the training set. In
addition, as they align the test image to all subjects
simultaneously, it potentially is more prone to local minima
when the number of subjects increases, as we will see in the
following experimental comparisons.
\begin{enumerate}
\item {\em Extended Yale B.} In this experiment, we have
    used the same experimental settings as in
    \cite{Huang2008-CVPR}. 20 subjects are selected and
    each has 32 frontal images (selected at random) as
    training and another 32 for testing. An artificial
    translation of 10 pixels (in both $x$ and $y$
    directions) is introduced to the test image. For our
    algorithm we downsample all the images to $88\times
    80$ for memory reasons, whereas the work of
    \cite{Huang2008-CVPR} uses random projections.
Note that the use of cropped images in this experiment introduces image boundary effects.
    Our
    algorithm achieves the recognition rate 93.7\%,
    compared to 89.1\% recognition rate reported in
    \cite{Huang2008-CVPR}.
\item {\em CMU Multi-PIE.} In this experiment, we choose
    all subjects from the CMU Multi-PIE database, 7
    training images from Session 1 and 1 test image from
    Session 2 per person. The setting is exactly the same
    as the previous experiment on 2D deformation. We again
    work with downsampled images of size $80\times 60$ pixels. An
    artificial translation of 5 pixels (in both $x$ and $y$
    directions) was induced in the test image. The
    algorithm of \cite{Huang2008-CVPR} achieves a
    recognition rate of 67.5\%,\footnote{That algorithm has
    two free parameters - $l$ and $d$, which govern the tradeoff between
	accuracy and run-time. For this experiment
    we chose $l = 1$ and $d = 514$.} while ours achieves 92.2\%.
    \end{enumerate}
				
\section{Handling Illumination Variation}\label{sec:illumination}
In the above section, we have made the assumption that the test image, although taken under some arbitrary illumination, can be linearly represented by a finite number of training illuminations.  Under what conditions is this a reasonable assumption to make?  What can we say from first principles about how the training images should be chosen?

\subsection{The Illumination Model}

The strongest theoretical results so far regarding the relationship
between illumination and the resulting sets of images is due to Basri and Jacobs \cite{Basri2003-PAMI}.
The main result of this section is that for convex Lambertian objects, distant illuminations, and fixed pose,
all images of the object can be well approximated by linear combinations of
nine (properly chosen) basis images.  The basis images have mixed sign, and
their illuminations consist of the lowest frequency spherical harmonics.
While this is a very important result for understanding the image
formation process, the direct application of this result in most practical
systems is misguided for several reasons.
Specularities, self-shadowing, and inter-reflections all dramatically affect the appearance of face images,
and they all do so in a way that violates the modeling assumptions of the Basri analysis.

Fortunately, even with these effects, for most materials the relationship between
illumination and image is still linear,\footnote{Materials that break
this assumption include fluorescent materials and the photochromic (``Transition'') lenses
in some eyeglasses.  Most materials emit light in proportion to their
incident light.} provided the sensor has a linear response curve.\footnote{Proper handling of gamma encoding is an important consideration for
practitioners.  Most cameras apply a non-linear and often undocumented response
curve to captured images.  A slight degradation of performance will occur if
gamma compressed images are treated as if they were linear.  We recommend the use
of cameras with well documented response curves that can be inverted when the
image file is loaded.}
For a more in-depth study
of the relationship between illumination and images, we refer the reader to
\cite{belhumeur1998set}.
While the relationship between illuminations and images is linear,
only positive weights are allowed; the space of all images of an object with
fixed pose and varying illumination is a convex cone lying in the positive
orthant. The question becomes, how many images does it take to do a good job
of representing images sampled from this cone?

It has been observed in various empirical studies that
one can get away with using a small number of frontal
illuminations to linearly represent a wide range of new frontal
illuminations, when they are all taken under the same laboratory conditions
\cite{Georghiades2001-PAMI}. This is the case for many public
face datasets, including AR, ORL, PIE, and Multi-PIE.
Unfortunately, we have found that in practice, a training
database consisting purely of frontal illuminations is not
sufficient to linearly represent images of a faces taken
under typical indoor or outdoor conditions (see the experiment
conducted in Section \ref{sec:own-data}). As illustrated by the
example in Figure \ref{fig:promo}, an insufficient number of
training illuminations can result in recognition failure. To
ensure our algorithm works in practice, we need to find a set
of training illuminations that are indeed {\em sufficient} to
linearly represent a wide variety of practical indoor and
outdoor illuminations.


\subsection{Projector-based Illumination System}

We have designed a system that can acquire frontal images of a subject while
simultaneously illuminating the subject from all directions above horizontal. A sketch of the
system is shown in Figure \ref{fig:system}: The illumination
system consists of four projectors that display various bright
patterns onto the three white walls in the corner of a dark
room.  The light reflects off of the walls and illuminates the
user's head indirectly.  After taking the frontal illuminations
we rotate the chair by 180 degrees and take pictures from the
opposite direction.  Having two cameras speeds the process
since only the chair needs to be moved in between frontal and
rear illuminations. Our projector-based system has several
advantages over flash-based illumination systems for face recognition:
\begin{itemize}
\item The illuminations can be modified in software, rather than hardware.
\item It is easy to capture many different illuminations quickly.
\item Good coverage and distant illumination can be achieved simultaneously.
\item There is no need to mount anything on the walls or construct a large dome.
\item The system can be assembled from off-the-shelf hardware.
\end{itemize}
\begin{figure}
\centering
\includegraphics[height=2.5in]{figures_pami/camera_rig.pdf}
\caption{\small{\bf Training acquisition system:} Four projectors and two cameras controlled by one computer.}
\label{fig:system}
\end{figure}
With our projector system, our choice of illuminations is
constrained only by the need to achieve a good
SNR,\footnote{Since illuminations with more pixels illuminated
will have a better SNR (provided they don't saturate), there is
an engineering tradeoff between the SNR and the number of
training images.} avoid saturation, and achieve a reasonably
short acquisition time.  Two simplifying assumptions that we
make are that every pixel is either turned fully on or off in
every illumination, and that the illuminated regions do not
overlap.

Assuming that each pixel is fully on or off enables us to guarantee
that each illumination image has the same overall intensity, merely
by guaranteeing that we illuminate the same number of pixels in each image.\footnote{Since DLP projectors may have dramatically different response
curves depending on the mode they are in, it is not advisable to simply normalize each illumination image by its mean.}
Since our algorithm depends only on  the
linearity between the illuminations and the images, and not on the
relative intensities of the illuminations, the designer has the freedom to choose the overall intensity of the illuminations
to prevent saturation or low SNR, in a sort of offline exposure control.

Assuming that the sequentially illuminated regions do not overlap results in a
set of training images that span a larger cone than a similar number of
overlapping regions.  This results in training images that require fewer
negative coefficients in $\x$ to represent test images under natural
illuminations.  The effect of negative coefficients in $\x$ appears to depend
partly on how the test images are taken and is still under study.

{\em Relationship to existing work:} Most light stages used for face recognition have
been constructed for the purpose of creating public data sets to study
illumination invariance \cite{Georghiades2001-PAMI, Gross2008-FGR}.  Many other
light stages have been used for computer graphics purposes
\cite{debevec2000acquiring, jones2005performance}.
The light source can be
moved around manually \cite{masselus2002free}, but this may result in poor
consistency of illuminations between users.  Structured light applications use projectors to
directly illuminate the face (or other object) \cite{zhang2002rapid} for 3D
reconstruction, but this is very disturbing to the user.
Y.\ Schechner \cite{schechner2007multiplexing}
studies techniques for multiplexing illumination that can dramatically reduce the noise
of the demultiplexed images for certain classes of objects and cameras.
While these techniques have not been incorporated into the current
system, they fit elegantly into our framework and will likely be used
in future implementations.  We stress that use of this multiplexing technique
is independent from the choice of original (directional) illuminations.

\subsection{Choice of Illumination Patterns}

We ran two experiments to guide our choice of illuminations for
our large-scale experiments:
\begin{enumerate}
\item {\em Coverage Experiment.} In the first experiment we
    attempt to determine what coverage of the sphere is
    required to achieve good interpolation for test images.
    The subject was illuminated by 100 (50 front, 50 back)
    illuminations arranged in concentric rings centered at
    the front camera.  Subsets of the training images were
    chosen, starting at the front camera and adding a ring
    at a time.  Each time a ring was added to the training
    illumination set, the average $\ell^1$ registration
    error (residual) for a set of test images taken under
    sunlight was computed and plotted in Figure
    \ref{fig:illumination-patterns}(a).  The more rings of
    training illuminations are added, the lower the
    representation error becomes, with diminishing returns.
\item {\em Granularity Experiment.} In the second
    experiment we attempt to determine how finely divided
    the illumination sphere should be.  At the first
    granularity level, the projectors  illuminate the
    covered area uniformly.  At each subsequent granularity
    level each illuminated cell is divided in two along its
    longer side but intensity doubled.  For each
    granularity level the average $\ell^1$ registration
    error is computed as in the coverage experiment and
    shown in Figure \ref{fig:illumination-sufficiency}(b).
    Again, diminishing returns are observed as more
    illuminations are added.
\end{enumerate}
\begin{figure}
\centering
\begin{tabular}{cc}
\includegraphics[height=1.7in]{figures_pami/coverage_experiment_asplode.png} &
\includegraphics[height=1.7in]{figures_pami/final_cvpr_illuminations_asplode.png}  \\
(a) Coverage Experiment & (b) Chosen Illumination Patterns
\end{tabular}
\caption{\small{\bf Illumination Patterns.}   The cells are illuminated in sequence.  For rear illuminations the sequence is reversed.  In the chosen pattern's rear illumination, the cells 1-5 and 7-11 are omitted for a total of 38 illuminations. The four rectangular regions correspond to the four projectors.  }
\label{fig:illumination-patterns}
\end{figure}

\begin{figure}
\centering
\begin{tabular}{@{}c@{}c@{}}
\includegraphics[height=2in]{figures_pami/illum_results/coverage_sunset.pdf} &
\includegraphics[height=2in]{figures_pami/illum_results/granularity_sunset.pdf} \\
(a) Coverage & (b) Granularity
\end{tabular}
\caption{\small{\bf Study of sufficient illuminations.} The average $\ell^1$ registration residual versus different illumination training sets. }
\label{fig:illumination-sufficiency}
\end{figure}

In the plot for the coverage experiment, Figure
\ref{fig:illumination-sufficiency}(a),
 we clearly see two plateau regions: one is after 4 rings
and one is after 10 rings. The first four rings represent the
typical frontal illuminations, which are present in most public
face datasets; however, we see that the residual stabilizes
after 10 rings which include some illuminations from the back
of the subject. This suggests that although the frontal
illuminations account for most of the illumination on the face,
some illuminations from the back are needed in the training set to
represent images with illumination coming from all directions.
In the plot for the granularity experiment, Figure
\ref{fig:illumination-sufficiency}(b), we observe that the
residual reaches a plateau after four divisions, corresponding
to a total of 32 illuminations. Based on the results from both
experiments, we decide to partition the area covered by the
first 10 rings into a total of 38 cells, whose layout is
explained in Figure \ref{fig:illumination-patterns}(b). For
our large-scale experiments, we have collected those
illuminations for all our subjects.\footnote{It is possible
that with further experimentation a reduced set of illuminations
can be found that performs as well or better.}

See below for the 38 training images of one subject:
\begin{figure}[h]
\centering
\includegraphics[width=\textwidth]{figures_pami/training.png}
\end{figure}


%\subsection{Should Positivity in the Representation Coefficients
%be Enforced?} One critical issue in linear illumination models
%is whether to enforce nonnegativity in the coefficients $\x$,
%i.e. whether to model illumination using a cone or a subspace.
%Some authors, i.e. \cite{Basri2003-PAMI}, have chosen to allow
%negative components in the coefficient vector $\x$ when
%representing one image as a sum of other images, and only
%enforce that the linear combination $A\x$ be positive
%everywhere.  Unfortunately, for non-convex objects such as
%faces, enforcing non-negativity of the representation is not
%sufficient to guarantee that the resulting image of the object
%is physical.  For example, consider a Lambertian planar object
%with a cylindrical hole drilled in it that is aligned with the
%optical axis of an orthographic camera.  Consider two images:
%one under illumination from a distant point source aligned with
%the hole, and one off of the hole axis.  Positively weighting
%the on-axis image and negatively weighting the on-axis image
%can result in an image where the bottom of the hole is more
%strongly illuminated than the surrounding plane.  The image is
%positive, but it is clearly non-physical.  It is worth
%emphasizing that this phenomenon is a result of some points on
%an object seeing a restricted subset of the illumination sphere
%(a violation of convexity),  and not a result of the
%directionality of the light chosen in the thought experiment.
%While the number of coefficients that actually end up negative
%after optimization is usually very small, we have observed
%pathological cases where the shadows next to the user's nose in
%the reconstructed image invert in value.

%Instead of using a smaller set of training images, allowing the
%coefficients to go negative to try to represent more test
%illuminations, and risking over-fitting, we decided to enforce
%non-negativity of the coefficients.  Nonnegative combinations
%of images are guaranteed to correspond to physically plausible
%images.  all physical illuminations unless the training images
%actually span the boundary of the illumination cone. Because we
%have a flexible acquisition system, we can directly generate a
%set of illuminations that span most of the illumination cone,
%without resorting to negative coefficients and risking
%overfitting.  Thus, in Algorithm 1, we have enforced that the
%coefficients $\x$ be non-negative.

%One of the main factors that complicates face recognition, and computer vision in general, is that two images of the same person's face can be very different, even if the pose is carefully controlled and all points on the person's face are visible in both images.  The most safe, reliable, and general assumption that can be made about the imaging process is that it is linear with respect to the intensity of different light sources.  The set of images of a given object is therefore a cone in the non-negative orthant of the pixel basis.  All recognition algorithms that rely on vision for training rely on images sampled from this cone.


\section{Tests on Public Databases}\label{sec:multipie}
In this section and the next section, we conduct comprehensive experiments on
large-scale face databases to verify the performance of our algorithm and
system. We first test on the largest public face database available that is
suitable for testing our algorithm, the CMU Multi-PIE.  One shortcoming of the
CMU Multi-PIE database for our purposes is that there is no separate set of
test images taken under natural illuminations; we are left to choose which sets
of images to use for testing and training.  To challenge our algorithm, we
choose only a small set of illuminations for the training set, yet we include
all illuminations in the testing set. In the following section, we will test
our algorithm on a face dataset that is collected by our own system. The goal
for that experiment will be to show that with a sufficient set of training
illuminations for each subject, our algorithm indeed works stably and robustly
with practical illumination, misalignment, pose, and occlusion, as already
indicated by our experiment shown in Figure \ref{fig:promo}(bottom).

CMU Multi-PIE provides the most extensive test set among public
datasets. This database contains images of 337 subjects across
simultaneous variation in pose, expression, and illumination.
Of these 337 subjects, we use all of the 249 subjects present
in Session 1 as the training set. The remaining 88 subjects are
treated as ``impostors'', or invalid images. For each of the
249 training subjects, we include frontal images of 7 frontal
illuminations,\footnote{They are illuminations
$\{0,1,7,13,14,16,18\}$ of \cite{Gross2008-FGR}. For each
directional illumination, we subtract the ambient-illuminated
image 0.} taken with neutral expression. As suggested by the
work of \cite{Georghiades2001-PAMI}, we choose these extreme
frontal illuminations in the hope that they would linearly
represent other frontal illuminations well. For the test set,
we use all 20 illuminations from Sessions 2-4, which were
recorded over a period of several months. The dataset is
challenging due to the large number of subjects, and due to
natural variation in subject appearance over time.
Table \ref{tab:MultiPIE-recognition2} shows the result of our
algorithm on each of the 3 testing sessions. Our algorithm
achieves recognition rates above $90\%$ for all three sessions.
For the test images, our iterative alignment was initialized
automatically via the Viola and Jones' face detector. To
demonstrate that the sparse representation based recognition
step is indeed beneficial even when there are no impostors, we
include results for recognition based only on the alignment
error residuals (i.e. $S=1$), shown in row 1.

\begin{table}
\caption{Recognition rates on the Multi-PIE database for
Algorithm 1 and \cite{Yang2010-CVPR}}
\centerline{
\begin{tabular}{|c|c|c|c|c| }
\hline
Recognition rate & Session 2 & Session 3 & Session 4  \\
\hline
{Alg. 1, $S=1$} & 90.7\% & 89.6\% & 87.5\% \\
\hline
{Alg. 1} & 93.9\% & 93.8\% & 92.3\% \\
\hline
{Alg. 1 with improved window} & 95.0\% & {\bf 96.3}\% & {\bf 97.3}\% \\
\hline
\cite{Yang2010-CVPR} & {\bf 95.2}\% & 93.4\% & 95.1\% \\
\hline
\end{tabular}
\label{tab:MultiPIE-recognition2} }
\end{table}

\newcommand{\tempwidth}[0]{0.9in}
\begin{figure}
\centering
{
\begin{tabular}{@{}c@{}c@{}c@{}c@{}c@{}c@{}}
\hspace{-2mm}\includegraphics[width=\tempwidth,clip=true]{figures_pami/multipie_failed/079_01_01_051_08.png}  &
\includegraphics[width=\tempwidth,clip=true]{figures_pami/multipie_failed/111_01_01_051_08.png}  &
\includegraphics[width=\tempwidth,clip=true]{figures_pami/multipie_failed/196_01_01_051_08.png}  &
\includegraphics[width=\tempwidth,clip=true]{figures_pami/multipie_failed/130_01_01_051_08.png}  &
\includegraphics[width=\tempwidth,clip=true]{figures_pami/multipie_failed/163_01_01_051_08.png}  &
\includegraphics[width=\tempwidth,clip=true]{figures_pami/multipie_failed/175_01_01_051_08.png} \\
\hspace{-2mm}\includegraphics[width=\tempwidth,clip=true]{figures_pami/multipie_failed/079_02_01_051_08.png}  &
\includegraphics[width=\tempwidth,clip=true]{figures_pami/multipie_failed/111_02_01_051_08.png}  &
\includegraphics[width=\tempwidth,clip=true]{figures_pami/multipie_failed/196_02_01_051_08.png}  &
\includegraphics[width=\tempwidth,clip=true]{figures_pami/multipie_failed/130_02_01_051_08.png}  &
\includegraphics[width=\tempwidth,clip=true]{figures_pami/multipie_failed/163_02_01_051_08.png}  &
\includegraphics[width=\tempwidth,clip=true]{figures_pami/multipie_failed/175_02_01_051_08.png} \\
\hspace{-2mm}(a) & (b) & (c) & (d) & (e) & (f) 
\end{tabular}
}
\caption{\small{\bf Representative failures from Multi-PIE}. {\bf Top:} training from Session 1; {\bf Bottom:} test images from Session 2. Due to changes in hair, glasses, beard, or pose, our alignment fails on these subjects regardless of test image illumination.}
\label{fig:failed-examples}
\end{figure}
\subsection{Improving the Sampling Window}
Our algorithm's errors are mostly caused by a few subjects who
significantly change their appearances between sessions (such
as hair, facial hair, and eyeglasses). Some representative
examples are shown in Figure \ref{fig:failed-examples}. For those subjects, alignment and recognition fail on
almost all test illuminations.
\begin{figure}[b]
\centering
{
\begin{tabular}{@{}cc@{}}
\includegraphics[trim=1.9in .7in 1.9in .5in, clip, height=1.8in]{figures_pami/example.png} &
\includegraphics[trim=1.9in .7in 1.9in .5in, clip, height=1.8in]{figures_pami/example_new.png} \\
Default window. & Proposed window.
\end{tabular}
}
\caption{\small{\bf Choosing different sampling windows.}}
\label{fig:new-mask}
\end{figure}
Meanwhile, this observation also suggests that we might be able
to improve the performance of our method by carefully choosing
a face region which is less affected by the above factors for
recognition. In particular, since the forehead region is likely
to be affected by the change of hair style, we try replacing
the previous $80 \times 60$ canonical frame with a new
window that better excludes the forehead. We adjust the
resolution of the window to keep $m$ approximately constant. In addition,
we cut off two lower corners of the $80 \times 60$ canonical frame, motivated by
the observation that in many cases the corners
actually contain background. An example of the new window
is shown in Figure~\ref{fig:new-mask}.

\begin{table*}
\centering
\small
\caption{\small Recognition rates on the Multi-PIE database for
different pairings of alignment and recognition stages.}
\centerline{
\begin{tabular}{|c|c|c|c|c|c|c|c|c|c|c| }
\hline
\backslashbox{Rec.}{Align.}
& \multicolumn{3}{|c|}{Face Detector}
& \multicolumn{3}{|c|}{Manual}
& \multicolumn{3}{|c|}{Iterative Alignment}
\\
\hline
Session $\rightarrow$	& 2		&3			&4			& 2		&3			&4			& 2		&3			&4		\\
\hline
NS	& 30.8\%	& 29.4\%	& 24.6\%	& 77.6\%	& 74.3\%	& 73.4\%	& 84.5\%	& 82.3\%	& 81.4\% \\
\hline
NN	& 26.4\%	& 24.7\%	& 21.9\%	& 67.3\%	& 66.2\%	& 62.8\%	& 73.5\%	& 69.6\%	& 69.3\% \\
\hline
LDA	& 5.1\%		& 5.9\%		& 4.3\%		& 49.4\%	& 44.3\%	& 47.9\%	& 91.0\%	& 89.9\%	& 88.1\% \\
\hline
LBP	& 39.9\%	& 38.1\%	& 33.9\%	& 93.3\%	& 91.2\%	& 92.9\%	& {\bf 95.2\%}	& {\bf 94.7\%}	& {\bf 93.5\%} \\
\hline
SRC	& -- & -- & -- & -- & -- & -- & 93.9\%	& 93.8\%	& 92.3\% \\
\hline
\end{tabular}
\label{tab:MultiPIE-recognition} }
\end{table*}

Table \ref{tab:MultiPIE-recognition2} shows that the
recognition rates on Multi-PIE indeed increase with this new
window. In addition, Figure \ref{fig:failed-examples}(a), (b),
and (c) illustrate three representative subjects for which the
recognition rates of our algorithm are significantly boosted
with the new window. However, we should mention that the best
choice of the window is problem-specific and there is not a
simple guideline to follow. For example, although the new
window performs better on Multi-PIE, the same window does not
help at all on our own database, which will be introduced in
the next section. This is because most of the training and
testing images in our database are taken on the same day so the
variation in hair style is very small. Hence, excluding the
forehead part may actually result in loss of useful
discriminative information.

\subsection{Comparison to Existing Work}

We first compare our result to the recent work
\cite{Yang2010-CVPR}. Notice that in \cite{Yang2010-CVPR}, the
initial registration is obtained from manually selected outer
eye corners. Then, a supervised hierarchical sparse coding
model based on local image descriptors is trained, which enjoys
certain translation invariant properties. With the same
training and testing sets, \cite{Yang2010-CVPR} is able to
handle the remaining misalignment and achieves state-of-the-art
performance on the CMU Multi-PIE database.
Table~\ref{tab:MultiPIE-recognition2} shows that our algorithm
achieves similar or better performance on different sessions of
Multi-PIE.

% Classical algorithms
To better examine the effectiveness of our iterative alignment
algorithm, we next compare our result to baseline
linear-projection-based algorithms, such as Nearest Neighbor
(NN), Nearest Subspace (NS) \cite{Lee2005-PAMI}, and Linear
Discriminant Analysis (LDA)
\cite{Belhumeur1997-PAMI}.\footnote{We do not list results on
PCA \cite{Turk1991-CVPR} as its performance is always below
that of Nearest Subspace.} Since these algorithms assume
pixel-accurate alignment, they are not expected to work well if
the test image is not well aligned with the training. In
Table~\ref{tab:MultiPIE-recognition}, we report the results of
these classical algorithms with three types of testing image
alignment: 1.\ alignment from the Viola and Jones' detector,
2.\ alignment via manually selected outer eye
corners,\footnote{Two manually clicked points are sufficient to
define a similarity transformation. All of the experiments in
this section are carried out with similarity transformations.}
and 3.\ the output of our iterative alignment algorithm. The
performance drop of the LDA algorithm on Multi-PIE reported
here seems to agree with that reported already in
\cite{Gross2008-FGR}.  All of the classical algorithms benefit
greatly from being paired with our iterative alignment
algorithm.

% LBP
We also compare our result to Local Binary Patterns (LBP)
\cite{Ahonen2006-PAMI}, a local appearance descriptor which is
able to capture fine details of facial appearance and texture.
Due to its robustness to variations in illumination, facial
expression, aging and other changes, LBP has achieved the
state-of-the-art face recognition performance in the scenario
when only one sample per person is used for training
\cite{Tan06facerecognition}. In this section, we follow the same
steps as in \cite{Ahonen2006-PAMI} to construct an LBP
descriptor for each training and testing sample. The $80\times
60$ face region is first divided into a regular $10\times 10$
grid of cells, each of size $8\times 6$ pixels. Within each
cell, the histogram of 59 uniform binary patterns is then
computed, where the patterns are generated by thresholding 8
neighboring pixels in a circle of radius 2 using the central
pixel value. Finally, the local histograms are concatenated to
produce the global descriptor vector. As suggested in
\cite{Ahonen2006-PAMI}, the recognition is performed using a
nearest neighbor classifier with Chi square distance as the
distance measure and we report the recognition rates with the
same three types of input as before.

As shown in Table~\ref{tab:MultiPIE-recognition}, although LBP
achieves competitive recognition rates given manually aligned
training and testing samples, demonstrating its robustness to
moderate misalignment, it still benefits from using the output
of our iterative alignment algorithm as the input. In addition,
like the other classical algorithms, the performance of LBP
degrades dramatically if it is applied directly to the output
of a face detector. This is notable given that LBP is often
applied without any special alignment in practice. Finally, we
attribute the improvement in performance of LBP over SRC in
this experiment to its robustness to illumination components
that cannot be linearly interpolated by the training set.

In addition, although our algorithm is not designed for
recognition when there is only a single gallery image per user,
we compare its performance with LBP within this setting for
completeness. For this experiment, we use the FERET dataset
\cite{phillips1998feret}, which contains five standard
partitions: `fa' is the gallery containing 1196 frontal images
of 1196 subjects, and `fb', `fc', `dup1' and `dup2' are four
sets of probe images. The testing sets differ from the training
in facial expression (`fb'), illumination (`fc'), aging (`dup1'
) and long aging (`dup2'). In fact, except for `fb', we notice
significant changes of illumination in all the other three test
sets. For the training, we again crop and normalize the face
region from each original image to an $80\times 60$ window
using manually marked eye coordinates \cite{Deng2010-PR}. In
Table~\ref{tab:FERET-recognition}, we report the performance of
our algorithm on the four test sets, with input directly
obtained from the Viola and Jones' detector. We also report the
performance of LBP with the same three types of input as before:
we use letters
``$d$'', ``$m$'', and ``$i$'' to indicate face detector, manual
alignment, and our iterative alignment algorithm, respectively.

As expected, our algorithm does not perform well except for
`fb', in which the illumination is similar to the training and
the mere variation in facial expression is handled well by the
sparse error model. For the other three test sets, our
algorithm fails because the illumination changes and other
variations seriously violate the assumptions of our method.
This also explains why LBP performs worse with our iterative
alignment algorithm, compared to manual alignment. On the other
hand, while LBP achieves the best recognition rates given
manually aligned training and testing samples, its performance
degrades drastically when the input is obtained directly from
the face detector. It is also worth noting that similar poor
performance of LBP, as well as other descriptors, has been
observed on the Labeled Face in the Wild (LFW) database, where
the training is uncontrolled and limited and the input is
directly obtained from the face detector \cite{Wolf2008-ECCV}.
\begin{table}
\caption{Performance on single gallery image FERET dataset}
\centerline{
\begin{tabular}{|c|c|c|c|c| }
\hline
Recognition rate \% & fb & fc & dup1 & dup2 \\
\hline
$LBP_d$ & 54.8 & 10.3 & 29.8 & 19.8 \\
\hline
$LBP_m$ & {\bf 96.6} & {\bf 58.8} & {\bf 71.6} & {\bf 61.5} \\
\hline
$LBP_i$ & 94.5 & 42.8 & 46.5 & 21.1 \\
\hline
{Alg. 1} & 95.2 & 28.4 & 46.1 & 20.3 \\
\hline
\end{tabular}
\label{tab:FERET-recognition} }
\end{table}

All of these experimental results confirm that
both illumination and alignment need to be simultaneously handled
well in order to achieve accurate face recognition, even when there is
no obvious occlusion or corruption in the test.

\subsection{Subject Validation}

We test the algorithms' ability to reject invalid images of the
88 subjects not appearing in the training database. As
mentioned before, the \emph{sparsity concentration index} (SCI)
is used as the outlier rejection rule. Given the sparse
representation $\x$ of a test image with respect to $K$
training classes, the SCI measures how concentrated the
coefficients are on a single class in the dataset and is
defined as in \cite{Wright2009-PAMI}:
\begin{displaymath}
\textup{SCI}(\x) \doteq \frac{K \cdot \max_i \|\delta_i(\x)\|_1 /
\|\x\|_1 - 1}{K - 1} \in [0,1] .
\end{displaymath}
It is easy to see that if $\textup{SCI}(\x) = 1$, the test
image is represented using images from one single subject
class; if $\textup{SCI}(\x) = 0$, the coefficients are spread
evenly over all classes. Thus, we can choose a threshold $t \in
[0,1]$ for the proposed method and accept a test image as valid
if $\textup{SCI}(\x) \geq t$, and otherwise reject it as
invalid. We compare this classifier to classifiers based on
thresholding the error residuals of NN, NS, LDA, and LBP.
\begin{figure}[t]
{
\centerline{
\begin{tabular}{@{}cc@{}}
\includegraphics[height=2.5in]{figures_pami/pami_roc_revision2} &
\includegraphics[height=2.5in]{figures_pami/pami_roc2} \\
(a) & (b) \\
\end{tabular}
}}
\caption{\small {\bf ROC curves} for subject validation on Multi-PIE database,
(a) for all algorithms with iterative alignment, and
(b) for the classical algorithms with manual alignment (indicated by a subscript ``m'').}\label{fig:roc-multipie}
\end{figure}

Figure \ref{fig:roc-multipie} plots the receiver operating
characteristic (ROC) curves, which are generated by sweeping
the threshold $t$ through the entire range of possible values
for each algorithm.\footnote{Rejecting invalid images not in
the entire database is much more difficult than deciding if two
face images are the same subject. Figure \ref{fig:roc-multipie}
should not be confused with typical ROC curves for face
similarity, e.g., \cite{PhillipsP2007}.} On the left we can see
that the SCI based recognition approach significantly
outperforms the other algorithms, including LBP, even when all
algorithms are coupled with our proposed iterative alignment.
In the right plot we again see that classical algorithms, and
even LBP, are very sensitive to alignment.  Similar contrasts
between our algorithm and baseline algorithms were also
observed for SRC in \cite{Wright2009-PAMI}, though on much
smaller datasets.

\subsection{Recognition with Synthetic Random Block Occlusion}

We further test the robustness of our $\ell^1$-norm based
algorithm to synthetic occlusion. We simulate various levels of
occlusion from 10\% to 50\% by replacing a randomly located
block of the face image with an image of a baboon, as shown in
Figure~\ref{fig:multipie-occ-rec}. In this experiment, to avoid
any other factors that may contribute to extra occlusion of the
face (such as the change of hair style), we choose illumination
10 from Session 1\footnote{This is the same session as the
training set.} as testing. The rest of the experimental setting
remains unchanged. The table in
Figure~\ref{fig:multipie-occ-rec} shows that our algorithm is
indeed capable of handling a moderate amount of occlusion. For
example, at 20\% occlusion, our algorithm still achieves 94.9\%
recognition rate.

\renewcommand{\tempwidth}{0.2\textwidth}
\begin{figure}
\centering
\begin{tabular}{@{}c@{}c@{}c@{}c@{}c@{}}
\includegraphics[width=\tempwidth,clip=true]{figures_pami/multipie_occ/occ10.png} &
\includegraphics[width=\tempwidth,clip=true]{figures_pami/multipie_occ/occ20.png} &
\includegraphics[width=\tempwidth,clip=true]{figures_pami/multipie_occ/occ30.png} &
\includegraphics[width=\tempwidth,clip=true]{figures_pami/multipie_occ/occ40.png} &
\includegraphics[width=\tempwidth,clip=true]{figures_pami/multipie_occ/occ50.png}  \\
\end{tabular}
\caption{\small{\bf Recognition under varying level of
random block occlusion.} The above row of images shows examples of occluded test images with occlusion level from 10\% to 50\%. Our method maintains high recognition rates up to 30\% occlusion:}
{
\begin{tabular}{|c|c|c|c|c|c| }
\hline
Percent occluded & 10\% & 20\% & 30\% & 40\% & 50\%  \\
\hline
Recognition rate & 99.6\% & 94.9\% & 79.6\% & 46.5\% & 19.8\% \\
\hline
\end{tabular}
}
\label{fig:multipie-occ-rec}
\end{figure}

\subsection{Recognition with Pose and Expression} We now run tests of
our algorithm on a subset of the images from Multi-PIE with pose and expression variation in the test set, although we do not model these variations explicitly.
Using the same training set as above, we test our algorithm on
images in Session 2 with $15^\circ$ pose, for all 20
illuminations. As expected, the recognition rate drops to 78.0\%. We also test our
algorithm on images in Session 3 with smile, again for all 20
illuminations. The recognition rate is 64.8\%. Of course, it is reasonable to expect that
the performance of our method will be significantly improved if pose and expression data
are available in the training.


\section{Tests on Our Own Database}\label{sec:own-data} Using the training acquisition
system we described in Section \ref{sec:illumination}, and shown in Figure
\ref{fig:system}, we have collected the frontal view of 109
subjects {\em without eyeglasses} under 38 illuminations shown
in Figure \ref{fig:illumination-patterns}. For testing our
algorithm, we have also taken 935 images of these subjects with
a different camera under a variety of practical conditions.

\subsection{Necessity of Rear Illuminations} To see how
training illuminations affect the performance of our algorithm
in practice, we now compare how well a few frontal
illuminations can linearly represent: 1. other frontal illuminations
taken under the same laboratory conditions, and 2. typical
indoor and outdoor illuminations. To this end, we use the face
database acquired by our system and use 7 illuminations per
subject as training. The illuminations are chosen to be similar
to the 7 illuminations used in the previous experiment on
Multi-PIE.\footnote{We use the illuminations $\{6, 9, 12,
13, 18, 21, 22\}$ shown in Figure
\ref{fig:illumination-patterns}(b) to mimic the illuminations
 $\{0, 1, 6, 7, 13, 14, 18\}$ in Multi-PIE.} We then test
our algorithm on the remaining $24 - 7 = 17$ frontal
illuminations for all the subjects. The recognition rate is
$99.8\%$, nearly perfect. We also test our algorithm on 310
indoor images and 168 outdoor images of these subjects taken
under a variety of lighting conditions (category 1 and 2
specified below), similar to the one shown in Figure
\ref{fig:promo}, and the recognition rates for indoor and
outdoor images drop down to $94.2\%$ and $89.2\%$,
respectively. This is a strong indication that
frontal illuminations taken under laboratory conditions
are insufficient for representing test images under typical indoor and
outdoor illuminations.

\renewcommand{\tempwidth}{0.1667\textwidth}
\begin{figure}
\centering
\begin{tabular}{@{}c@{}c@{}c@{}c@{}c@{}c@{}}
\includegraphics[width=\tempwidth,clip=true]{figures_pami/uiuc_example/normal_indoor/DSC_1318.JPG} &
\includegraphics[width=\tempwidth,clip=true]{figures_pami/uiuc_example/normal_indoor/DSC_1521.JPG} &
\includegraphics[width=\tempwidth,clip=true]{figures_pami/uiuc_example/normal_indoor/DSC_1673.JPG} &
\includegraphics[width=\tempwidth,clip=true]{figures_pami/uiuc_example/normal_indoor/DSC_1732.JPG} &
\includegraphics[width=\tempwidth,clip=true]{figures_pami/uiuc_example/normal_indoor/DSC_1941.JPG} &
\includegraphics[width=\tempwidth,clip=true]{figures_pami/uiuc_example/normal_indoor/DSC_3766.JPG} \\
\includegraphics[width=\tempwidth,clip=true]{figures_pami/uiuc_example/normal_outdoor/DSC_1574.JPG} &
\includegraphics[width=\tempwidth,clip=true]{figures_pami/uiuc_example/normal_outdoor/DSC_1622.JPG} &
\includegraphics[width=\tempwidth,clip=true]{figures_pami/uiuc_example/normal_outdoor/DSC_1641.JPG} &
\includegraphics[width=\tempwidth,clip=true]{figures_pami/uiuc_example/normal_outdoor/DSC_3522.JPG} &
\includegraphics[width=\tempwidth,clip=true]{figures_pami/uiuc_example/normal_outdoor/DSC_3707.JPG} &
\includegraphics[width=\tempwidth,clip=true]{figures_pami/uiuc_example/normal_outdoor/DSC_3772.JPG} \\
\includegraphics[width=\tempwidth,clip=true]{figures_pami/uiuc_example/glasses/DSC_1397.JPG} &
\includegraphics[width=\tempwidth,clip=true]{figures_pami/uiuc_example/glasses/DSC_1532.JPG} &
\includegraphics[width=\tempwidth,clip=true]{figures_pami/uiuc_example/glasses/DSC_1556.JPG} &
\includegraphics[width=\tempwidth,clip=true]{figures_pami/uiuc_example/glasses/DSC_1585.JPG} &
\includegraphics[width=\tempwidth,clip=true]{figures_pami/uiuc_example/glasses/DSC_1688.JPG} &
\includegraphics[width=\tempwidth,clip=true]{figures_pami/uiuc_example/glasses/DSC_4035.JPG} \\
\end{tabular}
\caption{\small{\bf Representative examples of categories C1-C3}. One row for each category.}\label{fig:examples1-3}
\end{figure}

\begin{figure}[t]
\centering
\begin{tabular}{@{}c@{}c@{}c@{}c@{}c@{}c@{}}
\includegraphics[width=\tempwidth,clip=true]{figures_pami/uiuc_example/sunglasses/DSC_1565.JPG} &
\includegraphics[width=\tempwidth,clip=true]{figures_pami/uiuc_example/sunglasses/DSC_3656.JPG} &
\includegraphics[width=\tempwidth,clip=true]{figures_pami/uiuc_example/sunglasses/DSC_3827.JPG} &
\includegraphics[width=\tempwidth,clip=true]{figures_pami/uiuc_example/sunglasses/DSC_4090.JPG} &
\includegraphics[width=\tempwidth,clip=true]{figures_pami/uiuc_example/sunglasses/DSC_4106.JPG} &
\includegraphics[width=\tempwidth,clip=true]{figures_pami/uiuc_example/sunglasses/DSC_4126.JPG} \\
\includegraphics[width=\tempwidth,clip=true]{figures_pami/uiuc_example/sunglasses_failed/DSC_1611.JPG} &
\includegraphics[width=\tempwidth,clip=true]{figures_pami/uiuc_example/sunglasses_failed/DSC_3528.JPG} &
\includegraphics[width=\tempwidth,clip=true]{figures_pami/uiuc_example/sunglasses_failed/DSC_3744.JPG} &
\includegraphics[width=\tempwidth,clip=true]{figures_pami/uiuc_example/sunglasses_failed/DSC_3995.JPG} &
\includegraphics[width=\tempwidth,clip=true]{figures_pami/uiuc_example/sunglasses_failed/DSC_4030.JPG} &
\includegraphics[width=\tempwidth,clip=true]{figures_pami/uiuc_example/sunglasses_failed/DSC_4095.JPG} \\
\end{tabular}
 \caption{\small{\bf Representative examples of category C4}. Top row: successful examples with our method using overlapping blocks. Bottom row: failures with our method using overlapping blocks.}\label{fig:examples4}
\end{figure}

\subsection{Large-Scale Test with Sufficient Training
Illuminations} Now we use all 109 subjects and 38 illuminations
in the training and test on 935 images taken under a variety of
practical illuminations and conditions. We have manually partitioned the test images into four main
categories:
\begin{description}
\item[C1:] 310 \emph{indoor} images of 72 subjects without
    eyeglasses, frontal view
    (Fig.~\ref{fig:examples1-3}, row 1).
\item[C2:] 168 \emph{outdoor} images of 48 subjects without
    eyeglasses, frontal view
    (Fig.~\ref{fig:examples1-3}, row 2).
\item[C3:] 211 images of 32 subjects with \emph{eyeglasses}
    (Fig.~\ref{fig:examples1-3}, row 3).
\item[C4:] 246 images of 56 subjects with \emph{sunglasses}
    (Fig.~\ref{fig:examples4}).
\end{description}
We apply Viola and Jones' face detector on these images and
directly use the detected faces as the input to our algorithm.
Table~\ref{tab:UIUC-recognition} reports the performance of our
algorithm on each category.
Since our focus is on face
recognition, the errors do not include failures of the face
detector on some of the more challenging images.
As one can see, our algorithm achieves higher than 95\%
recognition rates on categories 1-3. Furthermore, using the
full set of 38 illuminations indeed improves the performance of
our system under practical illumination conditions compared to
only using a small subset of 7 illuminations. However, the
performance dramatically drops when the faces are occluded by
various types of sunglasses, which could cover up to 40\% of
the entire face. Given the previous experimental results on
synthetic random block occlusions, and given that the
illuminations are more challenging, the result is not
surprising. In the next subsection, we will show how additional
assumptions can be used to improve the recognition performance.
\begin{table}[h]
\centering \caption{Recognition rates on our own
database.}
\begin{tabular}{|c|c|c|c|c| }
\hline
Test Category & C1 & C2 & C3 & C4  \\
\hline
\hline
Recognition Rate & 98.4\% & 95.8\% & 95.1\% & 40.9\% \\
\hline
\end{tabular}
\label{tab:UIUC-recognition}
\end{table}

\subsection{Improving the Performance with Occlusion using Overlapping Blocks}
A traditional approach to improve the performance of face
recognition under severe occlusion is to use subregions instead
the entire face as a whole. This idea has been explored in many
earlier works; see \cite{Pentland1994-CVPR, Wright2009-PAMI}
for examples. Since in most real world cases the occlusion is contiguous, it is reasonable to argue that a minority of the
subregions are likely to be affected by the occlusion. In this
section, we adopt the same idea and partition the face into four
overlapping blocks to better handle sunglasses. This
scheme is illustrated in Figure~\ref{fig:occ-block}. Notice
that in this example three out of the four blocks are partially
or almost completely occluded. In our experiment, each block is
of size $90\times 48$ and covers about two-fifths of the entire
face. The testing and training sets are partitioned in the same
way. We then independently apply
Algorithm~\ref{alg:deformable-src} and compute a sparse
representation after registration for each block independently
with respect to the training set. The recognition
results for individual blocks are then aggregated by voting.

\begin{figure}
\centering
\includegraphics[width=4in]{figures_pami/occ_block.png}
\caption{\small{\bf Using overlapping blocks to tackle contiguous occlusion.} (a) The test image, occluded by sunglasses. (b) The four overlapping blocks. (c) The sparse representation is calculated after alignment for each block independently. The red lines correspond to his true identity. (d) The true identity is successfully recovered by voting based on the SCI scores.}
\label{fig:occ-block}
\end{figure}

In this experiment, we found that the using the \emph{sparsity
concentration index} (SCI) scores for voting achieves higher
recognition rate than the residual measure used in Algorithm~\ref{alg:deformable-src}, on
category 4 (sunglasses) of our database. The recognition rate
is increased to 78.3\%, compared to 40.9\% obtained without
this partition scheme. This is another evidence of the superior
ability of SCI on subject validation, since a heavily occluded
block can be regarded as an outlier for recognition and should
be rejected while voting.

However, we should point out that a major problem with this
approach is that occlusion cannot always be expected to fall within
 any fixed partition of the face image. Therefore, the
proposed scheme should only be viewed as an example which shows
that the performance under occlusion can be boosted by
leveraging local information of a face as well as global information. We
leave the investigation of more general models (e.g., MRF \cite{ZhouZ2009}) for face
recognition with both misalignment and occlusion as an
interesting future work.

\section{Conclusion}\label{sec:conclusion}
Using a well-though-out combination of existing ideas
(iterative image alignment, $\ell_1$-error function, SRC, using projectors for
illumination), we have proposed a system for recognizing human faces
from images taken under practical conditions that is conceptually simple, well
motivated, and competitive with state-of-the-art recognition systems for access
control scenarios.

The system achieves extremely stable performance under
a wide range of variations in illumination, misalignment, and even under small amounts of
pose and occlusion. We achieve very good recognition performance on
large-scale tests with public datasets as well as our practical face
images, while using only frontal 2D images in the gallery and no
explicit 3D face model.
Our system could potentially be extended to better handle large pose
and expression, either by incorporating training images with different poses or
expressions or by explicitly modeling and compensating the associated deformations
in the alignment stage.

Another important direction for future
investigation is to extend the alignment algorithm to better
tackle contiguous occlusion. We have demonstrated that misalignment can be naturally handled within the
sparse representation framework. More complicated models for
spatial continuity, such as Markov random fields, have also
been successfully integrated into the computation of a sparse
representation of well-aligned test images
\cite{Cevher2008-NIPS, ZhouZ2009}. A unified approach
for face alignment and recognition in the presence of
contiguous occlusion remains an open problem.

