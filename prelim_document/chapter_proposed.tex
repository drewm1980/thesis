%!TEX root = thesis.tex
\chapter{PROPOSED WORK}
\label{chap:proposed}

\section{Introduction}
The previous two chapters covered work that I have already performed towards my Ph.D. in the area of face recognition.  In this chapter I propose the main components of plan to turn our face recognition system into a complete package that is both accurate and scalable.  I haven already demonstrated that state-of-the-art face recognition performance can be achieved through careful control of the training image acquisition process and the use of an algorithm that effectively models illumination, pose, and occlusion.  Unfortunately, while our CPU implementation already performs several times as fast as our MATLAB implementation, it currently still takes over a minute to process a single test image.  This is clearly far too slow to be useful for an access control system. Since this is the biggest bottleneck for adoption of the proposed face recognition technology, solving this problem will be a major component of the remaining work for my proposed Ph.D. thesis.  Another major component of my research will be a systematic study of the choice of training illuminations.  Having already implemented a novel projector based training image acquisition system, I have the unique chance to automate the optimization of the set of training illuminations.  Another major step will be to combine the Markov Random Field occlusion handling with our iterative alignment algorithm.  Finally, there are some improvements that could be made to the training acquisition hardware itself to improve training image acquisition.  

\section{Optimizing the Speed of the Recognition Algorithm}

%\subsection{Factors causing the current implementation to run slowly}
%\label{sec:factors}

Before any progress can be made on improving the speed of the algorithm, we must first understand why it currently runs so slowly. The main reason is that the sparsity based alignment and recognition algorithms detailed in the previous chapters use entire images as features.  Many face recognition algorithms either start with much smaller features to begin with (i.e. image patches, SIFT features), or project the images down to a much lower dimension as a first step.  Not only are we using relatively large training images in their natural basis, we are performing optimizations on them repeatedly in a loop, instead of just once, as is the case with Eigenfaces.   These optimization routines involve operations on many of these images at a time (for example, the recognition step currently uses 38 training images).  This is a very expensive algorithm in terms of the amount of data that must be handled simultaneously, and would have been considered very impractical to implement until rather recent improvements in computing hardware.  While we have an enormous amount of data that must be operated on, the simplicity of our choice of features has resulted in an algorithm that is extremely data parallel.  Table \ref{tab:breakdown} shows a breakdown of the execution time for the different parts of the algorithm.  The first thing to note is that the algorithm's execution time will have to be improved by a factor of about 100 in order to be useful for access control.  It is unreasonable to expect a user to wait for much more than a second while the system decided whether or not to unlock the door.  Due to the emergence of readily available massively parallel processors in the form of programmable GPU's, this factor of speed improvement is actually more reasonable than it sounds.  Similar speedups have actually been achieved for some other data parallel tasks when ported from the CPU to the CUDA architecture.  The next few subsections will outline some strategies for optimizing the individual steps.
\begin{table}[h]
\centering
\begin{tabular}{|c|c|}
\hline
Operation & Execution Time (seconds)\\
\hline
Loading the training database & 74\\
\hline
Per-user alignment & 130\\
\hline
Resampling the training images & 18\\
\hline
Final recognition step &137\\
\hline
\end{tabular}\vspace{2mm}
\caption{Execution timing for a database of 100 training users with 30 xga grayscale training images per user} \label{tab:breakdown}
\end{table}

%{\bf Loading the training database}  
\subsection{Loading the training database}  
The training database can be potentially quite large; for a database of 100 users, the full resolution training images take up 3 GB of data.  While his can be reduced significantly for the per-user alignment step (we only need the smaller re-sampled training images), we will need to load the full resolution images for the users that make it to the final recognition step, and this access must happen with very low latency.  Clever management of the loading of training images into memory will clearly be necessary.

One strategy for reducing the latency of the image loading is to use a technology called memory mapping.  This creates a mapping between the data in a file in the system (most of which will be resident on the hard drive) and the address space of the program.  The operating system's virtual memory system then handles the loading of data from disk into RAM as needed when the program access the corresponding addresses in its memory space.  This technique has the potential to kill several birds with one stone:
\begin{itemize}
\item Since there is one virtual memory system for all processes in the os, portions of the database that are used by multiple processes can share the data that has been loaded into RAM.  This can significantly reduce the memory footprint when multiple instances of the recognition system are running simultaneously.
\item It is much easier to load just the portion of each image file that is needed.  Resampling the training images will only require a subset of the pixels from the high resolution image.  On x86 systems the memory page size is 2KB.  For a grayscale xga image laid out linearly one disk this corresponds to two rows of the image.  In this case only the rows that overlap the user's face will get loaded from disk into memory.  This can be improved further by laying out the data so that each memory corresponds to a tile of about 45 x 45 pixels.  Then only the tiles that overlap the users' face will have to get loaded into memory.
\item There is a mechanism to hint to the operating system what data will be needed in advance so that is can be pre-loaded.  
\end{itemize} 
Another strategy for reducing the latency of the image loading is to try to predict which images will need to be loaded in advance of when they are used.  In particular, we have some information about which users will likely make the cut for the recognition step while per-user alignment is still being performed.  Any training user that is not in the top $S$ (currently set to 10) users aligned so far certainly need not be loaded into memory.  

%{\bf Per-user alignment}  
\subsection{Per-user alignment}  
Since this operation must be performed once per training user, the cost of this operation grows proportionally with the size of the training database.  For each user, the computation for this step involves repeatedly solving an l1 optimization problem and resampling filtered versions of the test image.  

One strategy for reducing the amount of time spent resampling(warping) the training images is to make use of the fact that we are resampling $N$ (= 38 training images currently) simultaneously with the same transformation.  We can use this fact in two ways: first, we only have to compute the mapping of the pixel locations once, and second, we can tile the data for all 38 images together to further reduce time spent accessing memory.   

%{\bf Resampling the training images}  
\subsection{Resampling the training images}  
Again, clever tiling and pre-fetching will be able to greatly reduce the time spent in this task.

%{\bf Final recognition step}  
\subsection{Final recognition step}
This bulk of the time spent in the final recognition step is in a large l1 optimization.  The matrices involved are bigger, but the l1 optimization problem only has to be solved once.  

The two l1 optimization problems are solved by re-casting them as linear programming problems.  The resulting linear programming problems can be solved efficiently using interior point methods.  The most expensive operation in the interior point method for these problem sizes is the computation of the step direction.  This requires a multiplication of several large matrices to compute the matrices for a much smaller linear system of equations.  This smaller linear system is then solved by minimizing the $\ell^2$ norm of the error.  Since all of these operations are taking place on very large arrays, these optimizations have the potential to map well to massively programming architectures, such as GPU's.  The massively parallel (capable of executing hundreds of threads concurrently) architecture that is currently the most promising is Nvidia's CUDA architecture, and development will center on this development system for the foreseeable future.  Performance gains of up to 100x over CPU implementations have been reported for algorithms that map well to the GPU.

An example of an operation in our algorithm that is very low hanging fruit for GPU implementation is the multiplication of $A^T * D * A$ where $A$ is extremely tall (5120 x 38) and $D$ is diagonal.  While a general GPU implementation of matrix multiplication is provided with the GPU, there are likely significant performance improvements to be gained from a custom multiplication routine.  Data can be shared between the left and right copies of A.  To conserve cache space $A$ can be stored as the original 8 bit data type that came from the image, along with a floating point scale factor for each column.  The fact that $D$ is diagonal also greatly reduces the amount of computation that need be done.  Since $A$ is extremely tall, the edge case (in computation and in memory loading) for  the long edge is much more important that for the short edge.  Furthermore, the prior knowledge that $D$ gets updated about 10 times as frequently as $A$ how best to handle memory transfers.

While the efficient implementation of the sparse representation algorithms we used has a limited interest from a theoretical standpoint, it is absolutely critical for enabling the use of these techniques for vision applications.  Sparse representation may one day be as important a building block for computer vision as the singular value decomposition is currently, but this cannot happen without efficient implementations that are tuned both for the application domain and the available hardware.  

\section{Optimizing the training illuminations}
To date, I have performed two experiments have been performed to guide the choice of training illuminations, the results of which were shown in Chapter \ref{chap:cvpr}.  One used rectangular partitions of the domain of the illumination space with different granularities.  This gave a rough idea of the number of training illuminations that would be necessary.  The second experiment used a partitions of the illumination domain that were rectangular in a polar coordinate system.  This gave a rough idea of the range of angles the training images should cover.  Our large training database was acquired by combining these two pieces of information.  While a study of the choice of training illuminations at this granularity is already unprecedented for face recognition, \footnote{ Several studies, including \cite{Basri2003-PAMI, LeeK2005-PAMI} have used images rendered from a 3D face model to optimize their choice of illumination.  Similar illumination studies have also been performed on other objects for graphics applications.} only a very small subset of the possible illuminations was explored compared to what the projectors are capable of.  For instance, how many of the frontal images are necessary?  Should some directions receive a higher density of illuminations than others? 

\subsection{Optimize illuminations within the set of training images we already have}
One option we have for optimizing our training illuminations is to use the training and testing databases we have already gathered, and to search for a lower dimensional subspace of illuminations that still has a good recognition rate.  The two main advantages of this strategy are that the recognition rate can be computed directly and used as a measure of the quality of the training images, and that the sizable database we have already gathered can be used.  The main disadvantage is that we will most likely have to trade some accuracy for speed, since we can only restrict the number of measurements the algorithm uses. \footnote{It could be possible for a subset of the illuminations is more discriminating than the full set; however this would be a very surprising result indeed!}

\subsection{Optimize illuminations with the illumination system in the loop}
A second option is to perform an optimization of the training image space with the image acquisition system in the loop.  The main advantage is that we can include any training images we want to in the optimizaztion.  The main disadvantage is that we cannot compute a recognition rate and instead have to resort to using representation error as a measure of the quality of the training images.  Why can't we just capture and store a the images generated by a complete basis for the space of illuminations, and then optimize over weighted combinations of them?  If a single projector pixel is illuminated at a time, it would take over 4 TB of data to store the remaining images, and take 28 hours to capture; this wouldn't be convenient, but it could be managed.  Unfortunately, there is a problem with this idea:  in every image you take, the actual signal would fall below the noise floor of the camera.  For this reason it is critical to capture images with a significant portion of the projector pixels illuminated at a time.  Keeping the real world in the loop solves this problem without resorting to arbitrarily enforcing a minimum block size.  Due to the extremely long time that the subject will have to remain motionless while the optimization runs, a movie-grade dummy head would be a good candidate for the training subject.  

This still leaves several interesting questions.  Even with an automated search, we are going to have to make some assumptions to reduce the search space.  Should we allow for pixels to be partially turned on, or should they be binary?  Should we require illuminations to have pixels that are all adjacent?  What metric should we use to measure the quality of the training images?  Ideally we'd be using a large database of real subjects, but that is not an option if we want to search over the full set of illuminations the projectors can generate.  How many training images should we allow?  How do we quantify the tradeoff between speed and accuracy?

There are few other things that make this experiment appealing.  Since the dummy head is inanimate, we will completely eliminate the influence of pose variation on the experiment, and all of our training images will already be perfectly aligned; the cameras can be manually positioned such that the frontal and rear illuminations match up perfectly. \footnote{Since we want to capture training illuminations from both sides, the dummy head can be mounted on a stepper-motor driven pivot for the purpose of rotating the dummy head by 180 degrees.  }

\subsection{Leverage color information to improve occlusion robustness}  Color information is unused for the current system.  This was a simplifying assumption that made implementation of the algorithm significantly easier and faster.  There are several different ways in which the algorithms presented in this thesis could be extended to handle color information.  One of the main reasons that color is not as important in face recognition as it is in some other vision applications is that the pixels in the image of a given person's face generally vary primarily in intensity.  Furthermore, the variation of skin tone from person to person usually varies less than the color variation resulting from the color of the illumination.  For these reasons, color may be especially useful for the improvement of occlusion handling, since occluding objects are likely to vary in color more than human faces do.  

