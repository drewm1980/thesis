%!TEX root = thesis.tex
%\chapter{FUTURE WORK}
%\label{chap:future}
%This chapter contains some additional ideas I have for improving our recognition system.  Some of them I may implement to strengthen my two recent conference papers, and some I may work on in collaboration with others.

%\section{Improving The Accuracy of the Recognition Algorithm}
%\label{sec:performance}
%In addition to improving the speed of the recognition algorithm and the quality of the training images, there is still plenty of work to be done in modifying the algorithm to improve recognition rate.

\section{Combining MRF recognition with alignment} Chapter \ref{chap:iccv} demonstrated that it is possible to improve recognition performance by leveraging the information that occlusions are typically spatially coherent.  However, all of the experiments were performed on images that had already been aligned.  Combining this technique with alignment is an important topic for further research, and my collaboration with Zihan Zhou will continue on this topic.  The first attempt at integrating the MRF extension with alignment will be a straightforward combination of the two ideas.  Alignment is achieved by including a linearization of the warped test image with respect to the transformation parameters in the L1 optimization.  MRF simply restricts the subset of pixels on which the L1 optimization is performed.  Therefore, these two ideas are neatly orthogonal from an implementation standpoint.  The challenging part will be to understand exactly how the alignment and occlusion rejection interact in the early iterations of the algorithm, where the estimates of both the occlusion and the alignment have not yet converged significantly.


\section{Improving the acquisition system hardware}
The current design has proven to be very satisfactory in a research environment where the subjects are very cooperative and make an effort to hold very still while the images are being taken.  Unfortunately, not all users of this technology will be so careful.  The best way to improve the quality of the training images is to increase the speed at which they are acquired.  The primary motive for this is to reduce the amount of movement of the user's head between consecutive training images.  There are several possible ways in which this goal can be pursued:
%{\bf Refinements to the current design}
\begin{itemize}
\item {\em Increase the image acquisition rate by upgrading the cameras.}  Some newer IEEE1394 cameras contain features that could facilitate this.  One is the ability to trigger off of an external signal while still interleaving image exposure with transfer of the previous frame over the bus.  With our current cameras transfer and integration is serialized.  
\item {\em Implementing some type automatic exposure control}  Some illuminations cause more light to fall on the subject's face than others.  Therefore, the optimal exposure time depends on which illumination is being displayed.  Automatic compensation for this could reduce the average exposure time without hurting the SNR.
\item {\em Reduce synchronization delays by modifying the projectors}  There is a clock signal that drives the switching of the Digital Micromirror Device in each projector.  Synchronizing the camera exposure with a whole number of projector frames would greatly increase acquisition time; currently we have to be very conservative with the amount of time a given illumination is displayed to prevent problems synchronizing with the camera.
\item {\em Reduce exposure time by modifying the projectors}  The only way to decrease the exposure time without hurting signal to noise ratio is to increase the intensity of the lighting.  Most DLP projectors achieve color by spinning a wheel with either a colored mirror or a transmissive color filters in front of the light source.  Since color is not needed for  the training image acquisition, it would be desirable to remove the color wheel, effectively turning the projector into a black and white projector with a higher maximum intensity.
\end{itemize}
Not all of the above strategies may turn out to be feasible; in particular, the last two ideas would require some degree of reverse engineering of proprietary hardware inside the projector.  While the modifications are conceptually very simple, there may be unforeseen difficulties depending on the design of the projector.  In the worst case, the manufacturer may have deliberately implemented features to prevent tampering.

%{\bf Radical departures from the current design}
%The training image acquisition system detailed in Chapter \ref{chap:cvpr} requires several reasonably high quality DLP projectors and cameras for its operation.  While this is an investment that a large institution using face recognition  for access control might be willing to make, it pretty much rules outs application in less mission critical applications. There are several substitute techniques that may be worth pursuing:
%\begin{itemize}
%\item {\em Replace the projectors with many smaller light sources}  The main disadvantage of this technique is that it necessitates either building a large structure to hold the light sources or mounting them on the walls and ceiling.  Either way, a network of wires will be required to power and the actuate the lights, which further complicates installation.
%\item {\em Replace the projectors with a moving light source}  The main reason the projectors are not very cost effective in this application is that most of the light they produce is wasted.  This dramatically increases the need for thermal design, reduces the life of the components, and increases the power usage.  There are a variety of possible designs that get around this including the use of a moving light source, moving light-directing mirrors, or an array of focused light sources.
%\item {\em Replace the projectors with ambient lighting}  One face recognition application that has already hit the market is for securing user login on laptops.  Unfortunately, most individual users won't have access to a system that takes many pictures of them under controlled lighting.  One potential solution to this could be to have the user turn on a single light in their room, and then turn around in a circle while holding their (camera equipped) laptop.  The biggest hurdle for this technique will be that the training images will need to be aligned, since the user's head will likely move significantly during use.
%\end{itemize} 

\section{Conclusion}
Accelerating the algorithm to the point where it can be used for access control will initially be the highest priority.  Almost all parts of the algorithm will need to be re-implemented to take advantage of hardware specific optimizations on a new architecture, and this will be a major undertaking.  However this is effort is necessary not only to demonstrate that the recognition system can be applied to access control, but also so that the L1 optimization routines we use can be useful to others in the vision community.  Furthermore a faster implementation will enable me to perform experiments in the optimization of the training illuminations at a granularity that has never been attempted before.  It will be very interesting to see how much a more thoroughly optimized set of training illuminations can improve the recognition rate, and what those illuminations look like.  If the optimization of the training illuminations reliably converges to a particular set of illuminations, this work could settle once and for all how training illuminations for face recognition should be selected.  The robustness of the recognition algorithm to occlusions can likely be improved by integrating our new MRF results with our existing iterative alignment algorithm, and the quality of the training images can be improved through improved hardware integration.  The overall goal of the project is to deliver a face recognition system that is easy to implement, is robust to illumination variation, mild occlusions, and mild pose variations, and is scalable up to hundreds of users.
