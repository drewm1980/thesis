\chapter{$\ell_1$-Minimization Techniques}
\label{chap:minimization}

\section{Introduction}
While the previous chapters were dedicated to presenting a design for a highly
robust recognition pipeline, the remaining chapters are dedicated to speed optimized
algorithms and implementations of the pipeline presented in Chapter
\ref{chap:pipeline}.  

% Interior Point Intro
Traditionally, $\ell_1$-minimization (a.k.a.
basis pursuit (BP)) has been formulated as a linear program
\cite{ChenS2001-SIAM}. 
Several variations of the solution are also well known
in optimization, including a noisy approximation via quadratic programming
called the LASSO \cite{TibshiraniR1996} and truncated Newton interior-point
method (TNIPM) \cite{KimS2007}.
Recently, a number of accelerated algorithms have been proposed that
explicitly take advantage of the special structure of $\ell_1$-minimization
problems \cite{LorisI2009,YangA2010-ICIP}. 
The first published prototype of the face recognition pipeline utilized a customized
interior point solver \cite{WagnerA2009-CVPR}.  


% ALM intro
One of the drawbacks of most interior-point methods for $\ell_1$-minimization
is that they require the solution sequence to follow an interior path via
gradient descent or conjugate gradient methods, which are computationally
expensive.  To mitigate these issues, an approach called \emph{Homotopy} has
been recently studied to accelerate the speed of $\ell_1$-minimization
\cite{OsborneM2000,EfronB2004,MalioutovD2005,DonohoD2006}.  Although Homotopy
can be shown to exactly estimate BP when the solution is sufficiently sparse
\cite{DonohoD2006}, the algorithm still involves computationally expensive
operations such as matrix-matrix multiplication and linear least-squares
problems with varying $A$ matrices.  To overcome this, a state-of-the-art
$\ell_1$-minimization solution based on \emph{augmented Lagrangian methods}
(ALM) \cite{BertsekasD2003,YangA2010-ICIP} has been developed.

The ALM algorithm belongs to a
category of approximate $\ell_1$-minimization solutions called \emph{iterative
shrinkage-thresholding} (IST) methods \cite{WrightS2008,BeckA2009}.  IST
algorithms mainly involve elementary operations such as vector algebra and
matrix-vector multiplication. Therefore, when the dimension of the problem
becomes high, IST-type algorithms are particularly suitable for hardware
systems with a high degree of concurrency. In \cite{YangA2010-ICIP}, the
authors showed that ALM is able to significantly improve the solver speed,
while achieving estimation accuracy competitive with other $\ell_1$-minimization
solutions. Therefore, we choose ALM as the core algorithm for
implementation of $\ell_1$-minimization in the parallel face recognition pipeline.


%The second published prototype of the face recognition pipeline, as presented in 
%Chapter \ref{chap:pipeline} and in \cite{WagnerA2011-PAMI}, achieved a significant
%increase in recognition speed by using

%To recap, the core of the recognition
%pipeline consisted of solving $\ell_1$ minimization problems with the following two forms:
%\begin{equation}
%\min_{\x, \e} \| \x \|_1 + \|\e\|_1 \quad \subj \quad \bb = A \x + \e.
%\label{eqn:l1min_denoise}
%%\tag{\ref{eqn:l1min_denoise}}
%\end{equation}
%\begin{equation}
%\min_{\x, \e} \|\e\|_1 \quad \subj \quad \bb = A \x + \e.
%%\label{eqn:l1min_denoise}
%%\tag{\ref{eqn:l1min_denoise}}
%\end{equation}


We investigate parallelization of a



Although ALM succeeded in improving the speed of the face recognition pipeline
\cite{WagnerA2011-PAMI}, due to the high per-class cost of the alignment step,
the recognition system still fails to achieve \emph{real-time} performance
against datasets of hundreds or thousands of subjects.  

We contend that ALM is a better choice for implementation on many-core CPUs and
GPUs than interior point methods. 

In the next chapter,
in addition to accelerating the generic
$\ell_1$-minimization objectives \eqref{eq:l1min} and \eqref{eq:l1min_denoise}, we 
also discuss how to efficiently accelerate the face alignment step
\eqref{eq:l1min_alignment} on multi-core CPUs and GPUs.



\section{$\ell_1$-Minimization via Interior Point Method}
%\section{An Interior Point Solver for Alignment}
\label{sec:ipm_derivation}

This section outlines the derivation of an interior point solver, for the following optimization problem:
\begin{equation}
\min_{\x,\e,\q}\;  \| \e \|_1 \quad \mathrm{subj} \quad \bb = A \x + B \q + \e, \quad \x \ge \0.
\end{equation}, which forms the core of the iterative alignment routine.
It is a slightly more general version of problem (\ref{eq:l1min_alignment}),
that constrains a subset of the coefficients ($\x$) to be non-negative. 
Here, $\x \in \Re^n$, $\q \in \Re^p$, $\bb, \e \in
\Re^m$, so $A \in \Re^{m \times n}$, $B \in \Re^{m\times p}$. Our development
will follow Chapter 11 of Boyd and Vandenbergh. We begin by recasting this as
an inequality-constrained linear program in the usual fashion:
\begin{eqnarray*}
\min \qquad 1^* \u \\
\subj \qquad -\x &\le& \0 \\
\bb - A\x - B\q &\le& \u \\
-\bb + A \x + B\q &\le& \u
\end{eqnarray*}
To the three inequality constraints associate Lagrange multiplier vectors $\blambda_{x,+} \in \Re^n$, $\blambda_{e,+}, \blambda_{e,-} \in \Re^m$. Here, we have simply eliminated the equality constraint. The combined primal variables are 
\begin{equation}
\tilde \x = [ \, \x^* \;\, \q^* \;\, \u^* \, ]^* \in \Re^{n+m+p}.
\end{equation}
The objective is
\begin{equation}
\f^* \tilde \x \qquad   \f = [ \0_n \; \0_p \; \1_m ]^* \in \Re^{n+m+p}.
\end{equation}
The combined inequality constraint matrix 
\begin{equation}
\Phi \doteq \left[ \begin{array}{ccc} 
-\Id_{n \times n} & 0_{n \times p} & 0_{n \times m} \\
-A &  -B & - \Id_{m \times m} \\
A  &   B  &  - \Id_{m \times m} \\
\end{array} \right] \in \Re^{2m + n \times m + n + p}.
\end{equation}
Define 
\begin{equation}
\g \doteq \left[ \, \0_n \;\; \bb^* \;\, -\bb^* \, \right]^* \in \Re^{n + 2m}.
\end{equation}
The inequality constraint is then $$\Phi \tilde \x + \g \le \0_{n+2m}.$$
The combined dual variables are 
\begin{equation}
\blambda = [ \blambda_{x} \; \blambda_{e,+} \; \blambda_{e,-} ]^* \in \Re^{n+2m}.
\end{equation}
The KKT equations are 
\begin{eqnarray}
\Phi \tilde \x + \g &\le& \0_{n+2m} \\
\blambda &\ge& \0_{n+2m} \\
\f + \Phi^* \blambda &=& \0_{n+p+m} \\
\diag(\blambda) \,( \Phi \, \tilde \x + \g) &=& \0_{n + 2m}.
\end{eqnarray}
As usual in barrier methods, we take a $t^{-1}$-relaxation of the complementarity constraint. The relaxed KKT residuals are 
\begin{eqnarray}
\r_{dual}  &=&  \f + \Phi^* \blambda \\
\r_{central} &=& -\diag(\blambda) (\Phi \tilde \x + \g) - t^{-1} \1_{n+p+m}
\end{eqnarray}
The Newton system is then simply
\begin{equation}
\left[ \begin{array}{cc} 0 & \Phi^* \\ - \diag(\blambda) \Phi & -\diag(\Phi \tilde \x + \g) \end{array} \right] \left[ \begin{array}{c} \Delta \tilde \x \\ \Delta \blambda \end{array} \right] = - \left[ \begin{array}{c} \r_{dual} \\ \r_{central} \end{array} \right].
\end{equation}
This system has a lot of structure, so simplifications on paper are possible. Let
$$\Gamma \doteq - \diag(\Phi \tilde \x + \g).$$
The top $n+p+m$ equations involve only the dual variables. Expanding them,
\begin{equation} \label{eqn:d-lambda-only}
\left[ \begin{array}{ccc} -\Id_{n\times n} &  -A^* & A^* \\  0_{p \times n} & -B^* & B^* \\ 
0_{m \times n} & -\Id_{m \times m} & -\Id_{m \times m} \end{array} \right] \left[ \begin{array}{c} \Delta \blambda_{x} \\ \Delta \blambda_{e,+} \\ \Delta \blambda_{e,-} \end{array} \right] = - \left[ \begin{array}{c} \r_{d,1} \\ \r_{d,2} \\ \r_{d,3} \end{array} \right]
\end{equation}
where 
\begin{eqnarray*} 
\r_{d,1} &=& \r_{dual}(1:n) \in \Re^n \\
\r_{d,2} &=& \r_{dual}(n+1:n+p) \in \Re^p \\
\r_{d,3} &=& \r_{dual}(n+p+1:n+p+m) \in \Re^m \\
\end{eqnarray*} 
Immediately, we see that 
\begin{eqnarray}
\Delta \blambda_{e,-} &=& \r_{d,3} - \Delta \blambda_{e,+}.
\end{eqnarray}
Substituting into the top two equations of \eqref{eqn:d-lambda-only}, we have
\begin{eqnarray}
-\Delta \blambda_{x} - 2 A^* \Delta \blambda_{e,+} &=& - \r_{d,1} - A^* \r_{d,3} \\
-2 B^* \Delta \blambda_{e,+} &=& -\r_{d,2} - B^* \r_{d,3}
\end{eqnarray}
yielding
\begin{equation}
\Delta \blambda_{x} = -2 A^* \Delta \blambda_{e,+} +  \r_{d,1} + A^* \r_{d,3} .
\end{equation}
If you trace back all of the substitutions, what is left is the following system of equations:
\begin{eqnarray*}\small
B^* \Delta \blambda_{e,+} = \frac{1}{2} ( \r_{d,2} + B^* \r_{d,3} ) \;\doteq\; \a \\
\hspace{-.75in}\left[ \begin{array}{cccc} \Lambda_{x} & 0 & 0 & -2 \ \Gamma_{x} A^* \\ \Lambda_{e,+}A & \Lambda_{e,+} B & \Lambda_{e,+} & \Gamma_{e,+} \\ 
-\Lambda_{e,-} A& -\Lambda_{e,-} B & \Lambda_{e,-} & -\Gamma_{e,-} \end{array} \right] \left[ \begin{array}{c} \Delta \x \\ \Delta \q \\ \Delta \u \\ \Delta \blambda_{e,+} \end{array} \right] \\= \left[ \begin{array}{c} -\Gamma_{x} (\r_{d,1} + A^*\r_{d,3}) -\r_{c,1} \\ -\r_{c,2} \\ -\r_{c,3}-\Gamma_{e,-} \r_{d,3} \end{array} \right]
\end{eqnarray*}
The above manipulations have exposed some structure in the problem that we can take advantage of: the $\Lambda$ and $\Gamma$ are diagonal, and therefore easy to invert. The subtlety is that for each equation $j$, exactly one of $\Lambda_j$ and $\Gamma_j$ is converging to zero. To continue analytically eliminating terms, we need to choose ``well-conditioned subsets'' for each of these matrices. 

\paragraph{Eliminating $\Delta \u$.}

From $[2m]$, select one element from each $\{ i, i+m\}$ (with ties broken at random):
$$L^e_g \doteq \{ i : | [\blambda_{e,+} \; \blambda_{e,-}]_i | > | [\blambda_{e,+} \\ \blambda_{e,-} ]_{i+m} | \}.$$
Let $\Lambda_g$ be the diagonal matrix such that
$$\Lambda_g(i,i) = \left\{ \begin{array}{cc} \Lambda_{e,+}(i,i) & i \in L^e_g \\ \Lambda_{e,-}(i,i) & else \end{array}\right .$$ 
Let $\Sigma_g \in \Re^{n \times n}$ be the diagonal matrix such that 
$$\Sigma_g(i,i) = \left\{ \begin{array}{cc} 1 & i \in L^e_g \\ -1 & else \end{array}\right.$$
$$M_g(i,i) = \left\{ \begin{array}{cc} \Gamma_{e,+}(i,i) & i \in L^e_g \\ -\Gamma_{e,-}(i,i) & else \end{array}\right.$$ 
Let $\Lambda_b$, $\Sigma_b$, $M_b$ be defined in the natural (complement) manner. Define $\ggamma =  \left[ \begin{array}{c} - \r_{c,2} \\ -\r_{c,3} - \Gamma_{e,-} \r_{d,3} \end{array} \right]$, $\ggamma_g = \ggamma(L^e_g)$, $\ggamma_b = \ggamma(L^e_b)$.

Then again we have two equations:
\begin{eqnarray}
\Sigma_g \Lambda_g A \Delta \x \; + \Sigma_g \Lambda_g B \Delta \q + \Lambda_g \Delta \u + M_g \Delta \blambda_{e,+} &=& \ggamma_g \\
\Sigma_b \Lambda_b A \Delta \x \; +  \Sigma_b \Lambda_b B \Delta \q + \Lambda_b \Delta \u + M_b \Delta \blambda_{e,+} &=& \ggamma_b. \label{eqn:bot-bad}
\end{eqnarray}
Eliminate $\Delta \u$ via the good part:
\begin{equation}
\Delta \u = - \Sigma_g A \Delta \x - \Sigma_g B \Delta \q - \Lambda_g^{-1} M_g \Delta \blambda_{e,+} + \Lambda_g^{-1} \ggamma_g
\end{equation}
leaving 
\begin{equation}
2 \Sigma^e_b \Lambda^e_b A \Delta \x + 2 \Sigma^e_b \Lambda^e_b B \Delta \q + (M^e_b - \Lambda^e_b (\Lambda^e_g)^{-1} M^e_g) \Delta \blambda_{e,+} = \ggamma_b - \Lambda^e_b (\Lambda^e_g)^{-1} \ggamma_g.
\end{equation}
$\Psi \doteq M_b^e - (\Lambda^e_g)^{-1} \Lambda^e_b M^e_g$ is a well-conditioned diagonal matrix, allowing us to next eliminate $\Delta \blambda_{e,+}$:
\begin{equation}
\Delta \blambda_{e,+} = P\Delta x + T \Delta \q + \h,
\end{equation}
where 
\begin{equation}
P \doteq - 2 \Psi^{-1} \Sigma^e_b \Lambda^e_b A \qquad T \doteq -2 \Psi^{-1}\Sigma^e_b \Lambda^e_b B \qquad \h \doteq \Psi^{-1}( \ggamma_b - \Lambda^e_b (\Lambda^e_g)^{-1} \ggamma_g ).
\end{equation}

\paragraph{Core system of equations.} All of the above manipulation leaves us
with our core linear system of equations in unknowns $\Delta \x, \Delta \q$ ($n
+ p$ unknowns total): \begin{eqnarray} (W+FP) \Delta \x + F T \Delta \q &=& \v
- F \h \\ B^* P \Delta \x + B^* T \Delta \q &=& \a - B^* \h .  \end{eqnarray}
Thus, through algebraic manipulation and re-ordering of the data, the linear
system that must be solved to compute the updates for $\x$ and $\q$ reduced to
size $(n+p)$.  Unfortunately, having to solve this system of equations each
time the step direction is computed results in an algorithm that has a high
cost per iteration.  Fortunately, the algorithm converges in a relatively 
small number of iterations.  In contrast, the ALM algorithm derived in the next
section is significantly cheaper per iteration, but requires more iterations.


\section{Derivation of the Augmented Lagrange Multiplier (ALM) Method for $\ell^1$-norm minimization}


\subsection{Error correction}

The problem at hand is 
\begin{equation}
\min_{\x,\e} \, \|\x\|_1 + \|W\e\|_1 \quad \mathrm{s.t.} \quad \y = A\x + \e,
\end{equation}
where $\x \in Re^n$, $\e \in Re^m$, and $W \in Re^{m \times m}$ is a fixed diagonal matrix with non-negative entries. 
\smallbreak
We set $f(\x,\e) = \|\x\|_1 + \|W\e\|_1$, and $h(\x,\e) = \y - A\x - \e$. Clearly, $f$ and $h$ are convex functions in $\x$ and $\e$. In the ALM formulation, the basic steps in each iteration can be summarized as follows:
\begin{equation}
\left \{ 
\begin{array}{lll}
(\x_{k+1},\e_{k+1}) & = & \arg\min_{\x,\e} \, \|\x\|_1 + \|W\e\|_1 + \langle \bl_k, \y - A\x - \e \rangle + \frac{\mu_k}{2}\|\y - A\x - \e\|_2^2 \\
\bl_{k+1} & = & \bl_k + \mu_k (\y - A\x_{k+1} - \e_{k+1}) \\
\mu_{k+1} & = & \rho\cdot\mu_k
\end{array} 
\right . ,
\end{equation}
where $\bl_k$'s represent the Lagrange multipliers, and $\{\mu_k\}$ is a monotonically increasing positive sequence ($\rho > 1$).
\smallbreak
We focus our attention on the non-trivial first step:
\begin{equation}
\min_{\x,\e} \, \|\x\|_1 + \|W\e\|_1 + \langle \bl_k, \y - A\x - \e \rangle + \frac{\mu_k}{2}\|\y - A\x - \e\|_2^2.
\label{eqn:alm}
\end{equation}
Since the cost function is convex in $(\x,\e)$, the optimality conditions can be written as
\begin{equation}
\begin{array}{lll}
\p - A^T\bl_k + \mu_k\,A^T(A\x + \e - \y) & = & \mathbf{0}, \\
W\q - \bl_k + \mu_k\, (A\x + \e -\y) & = & \mathbf{0},
\end{array}
\end{equation}
where $\p \in \partial \|\x\|_1$, and $\q \in \partial \|\e\|_1$. Since it is difficult to solve these equations to obtain a closed-form solution for the optimal $\x$ and $\e$, we attempt to minimize \eqref{eqn:alm} by an iterative procedure. 
\smallbreak
Instead of minimizing wrt $\x$ and $\e$ simultaneously, we adopt an alternating strategy to solve \eqref{eqn:alm} as follows:
\begin{equation}
\left \{
\begin{array}{lll}
\e_{j+1} & = & \arg \min_{\e}  \|W\e\|_1 + \langle \bl_k, \y - A\x_j - \e \rangle + \frac{\mu_k}{2}\|\y - A\x_j - \e\|_2^2 \\
\x_{j+1} & = & \arg \min_{\x} \|\x\|_1 + \langle \bl_k, \y - A\x - \e_{j+1} \rangle + \frac{\mu_k}{2}\|\y - A\x - \e_{j+1}\|_2^2
\end{array}
\right .
\label{eqn:alt}
\end{equation}
Before discussing the solution to the above iterations, we introduce the soft-thresholding (or shrinkage) operator for scalars as follows:
\begin{equation}
\shrink(x,\alpha) = \sign(x)\cdot \max\{|x| - \alpha, 0\},
\end{equation}
where $\alpha > 0$.\footnote{If $\alpha = 0$, then the $\shrink(.)$ operator reduces to the identity operator.} When applied to vectors or matrices, the $\shrink(.)$ operator acts elementwise. 
\smallbreak
Now, the first step in \eqref{eqn:alt} has the following closed-form solution:
\begin{equation}
\,[\e_{j+1}]_i  =  \shrink \left(\left[ \y + \frac{1}{\mu_k}\bl_k - A\x_j \right] _i, \frac{W_{ii}}{\mu_k}\right), \quad i = 1,2,\ldots,m,
\end{equation}
where $[\e]_i$ represents the $i$th component of $\e$.
\smallbreak
It is tough to come up with a closed-form expression for the solution to the second step in \eqref{eqn:alt}. So, we solve it by yet another iterative procedure.\footnote{We basically use the APG to solve this.} Given $\x_k$ close to $\x$, we construct a first-order approximation to the quadratic term in the cost function in \eqref{eqn:alm} as follows:
\begin{equation}
\|\y - A\x - \e\|_2^2 \approx \|\y - A\x_k - \e\|_2^2 + \langle 2A^T(A\x_k+\e-\y), \x - \x_k \rangle + \tau\|\x-\x_k\|_2^2,
\end{equation}
where $\tau \geq \lambda_\mathrm{max} (A^TA) = \sigma^2_\mathrm{max}(A)$. With this approximation, the second step of \eqref{eqn:alt} can be solved iteratively using soft-thresholding as derived below:
\begin{IEEEeqnarray}{rCl}
& & \arg \min_\x \, \|\x\|_1+ \langle \bl_k, \y - A\x - \e \rangle + \frac{\mu_k}{2} \left(\|\y - A\x_k - \e\|_2^2 + \langle 2A^T(A\x_k+\e-\y), \x - \x_k \rangle + \tau\|\x-\x_k\|_2^2 \right) \IEEEeqnarraynumspace \\
& = & \arg \min_\x \, \frac{2}{\mu_k}\|\x\|_1 + \left \langle 2A^T\left(A\x_k + \e - \y - \frac{1}{\mu_k}\bl_k\right), \x \right \rangle + \tau\|\x-\x_k\|_2^2 \\
& = & \arg \min_\x \, \frac{2}{\mu_k}\|\x\|_1 - \left \langle 2\tau \x_k - 2A^T\left(A\x_k + \e - \y - \frac{1}{\mu_k}\bl_k\right), \x \right \rangle + \tau\|\x\|_2^2 \\
& = & \arg \min_\x \, \frac{1}{\mu_k\tau}\|\x\|_1 - \left \langle \x_k - \frac{1}{\tau}A^T\left(A\x_k + \e - \y - \frac{1}{\mu_k}\bl_k\right), \x \right \rangle + \frac{1}{2}\|\x\|_2^2 \\
& = & \arg \min_\x \, \frac{1}{\mu_k\tau}\|\x\|_1 + \frac{1}{2}\left \| \x - \left(\x_k - \frac{1}{\tau}A^T\left(A\x_k + \e - \y - \frac{1}{\mu_k}\bl_k\right) \right)\right \|_2^2 \\
& = & \shrink\left(\x_k - \frac{1}{\tau}A^T\left(A\x_k + \e - \y - \frac{1}{\mu_k}\bl_k\right), \frac{1}{\mu_k\tau}\right)
\end{IEEEeqnarray}
\smallbreak
Thus, the iteration to solve the second step of \eqref{eqn:alt} can be written as:
\begin{equation}
\x_{l+1}  =  \shrink\left(\x_l - \frac{1}{\tau}A^T\left(A\x_l + \e_{j+1} - \y - \frac{1}{\mu_k}\bl_k\right), \frac{1}{\mu_k\tau}\right).
\end{equation}
Thus, the entire algorithm can be summarized as Algorithm \ref{alg:exact}.
\begin{algorithm}[h]
\caption{Exact ALM}
\begin{algorithmic}
\WHILE{not converged}
\WHILE{not converged}
\STATE $[\e_{j+1}]_i  =  \shrink \left(\left[ \y + \frac{1}{\mu_k}\bl_k - A\x_j \right] _i, \frac{W_{ii}}{\mu_k}\right), \quad i = 1,2,\ldots,m$
\STATE $t_1 \leftarrow 1$, $\z_1 \leftarrow \x_j$
\WHILE{not converged}
\STATE $\x_{l+1}  =  \shrink\left(\z_l - \frac{1}{\tau}A^T\left(A\z_l + \e_{j+1} - \y - \frac{1}{\mu_k}\bl_k\right), \frac{1}{\mu_k\tau}\right)$
\STATE $t_{l+1} = 0.5\left( 1 + \sqrt{1+4t_l^2}\right)$
\STATE $\z_{l+1} = \x_{l+1} + \frac{t_l - 1}{t_{l+1}}(\x_{l+1} - \x_l)$
\ENDWHILE
\ENDWHILE
\STATE $\bl_{k+1} = \bl_k + \mu_k (\y - A\x_{k+1} - \e_{k+1})$
\STATE $\mu_{k+1} = \rho\cdot\mu_k$
\ENDWHILE
\end{algorithmic}
\label{alg:exact}
\end{algorithm}
\smallbreak
Algorithm \ref{alg:exact} is not very fast in practice. This is mainly because the inner loops can take a lot of iterations to converge. In theory, any value of $\rho > 1$ ensures convergence. In practice, a large value of $\rho$ implies that the number of outer loops is small, but the number of inner loops increases dramatically. Besides, in practice, each inner loop must be terminated based on some kind of a tolerance criterion. The higher the tolerance, the less accurate the solution.
\smallbreak
{\bf Moving from exact to inexact.} To speed up the algorithm, we solve \eqref{eqn:alm} approximately, not exactly. This is akin to solving the inner loops in Algorithm \ref{alg:exact} with high tolerance or equivalently, limiting the number of iterations that each inner loop can take. As observed above, the extent of approximation depends on $\rho$. If $\rho$ is very close to 1, then limiting each inner loop to a small number of iterations does not hurt too much. However, choosing $\rho$ close to 1 would result in a large number of outer loop iterations. This is the major {\bf trade-off} in tuning the algorithm. The actual values chosen would depend on the relative speeds of matrix-vector multiplications and shrinkage operations. 
\smallbreak
Typically, on a single-core MATLAB implementation, if $\rho$ is close to 1, most of the time is spent on matrix-vector multiplications. If $\rho$ is large, then the shrinkage operator in the innermost loop is used very frequently, and the time associated with it (including reading in vectors, etc) becomes significant. 
\smallbreak
{\bf YALL1 package.} We have been using this MATLAB package so far. From what I understood from their paper, they consider the extreme approximation by limiting each inner loop to just 1 iteration. My hunch is this can be made to work if $\rho$ is chosen very close to 1 (say around 1.005-1.01), thereby allowing the number of outer loop iterations to increase a lot. 

\subsection{Compressed sensing}

The problem at hand is 
\begin{equation}
\min_{\x} \, \|\x\|_1 \quad \mathrm{s.t.} \quad \y = A\x,
\end{equation}
where $\x \in Re^n$ and $A \in Re^{m\times n}$. 
\smallbreak
We set $f(\x) = \|\x\|_1$, and $h(\x) = \y - A\x$. Clearly, $f$ and $h$ are convex functions in $\x$. In the ALM formulation, the basic steps in each iteration can be summarized as follows:
\begin{equation}
\left \{ 
\begin{array}{lll}
\x_{k+1} & = & \arg\min_{\x} \, \|\x\|_1 + \langle \bl_k, \y - A\x \rangle + \frac{\mu_k}{2}\|\y - A\x\|_2^2 \\
\bl_{k+1} & = & \bl_k + \mu_k (\y - A\x_{k+1}) \\
\mu_{k+1} & = & \rho\cdot\mu_k
\end{array} 
\right . ,
\end{equation}
where $\bl_k$'s represent the Lagrange multipliers, and $\{\mu_k\}$ is a monotonically increasing positive sequence ($\rho > 1$).
\smallbreak
We focus our attention on the non-trivial first step:
\begin{equation}
\min_{\x} \, \|\x\|_1 + \langle \bl_k, \y - A\x \rangle + \frac{\mu_k}{2}\|\y - A\x\|_2^2.
\label{eqn:alm_cs}
\end{equation}
Since the cost function is convex in $\x$, the optimality conditions can be written as
\begin{equation}
\p - A^T\bl_k + \mu_k\,A^T(A\x - \y) = \mathbf{0}
\end{equation}
where $\p \in \partial \|\x\|_1$. Since it is difficult to solve this equation to obtain a closed-form solution for the optimal $\x$, we attempt to minimize \eqref{eqn:alm_cs} by an iterative procedure. 
\smallbreak
It is tough to come up with a closed-form expression for the solution to the second step in \eqref{eqn:alt}. So, we solve it by yet another iterative procedure.\footnote{We basically use the APG to solve this.} Given $\x_k$ close to $\x$, we construct a first-order approximation to the quadratic term in the cost function in \eqref{eqn:alm} as follows:
\begin{equation}
\|\y - A\x\|_2^2 \approx \|\y - A\x_k\|_2^2 + \langle 2A^T(A\x_k-\y), \x - \x_k \rangle + \tau\|\x-\x_k\|_2^2,
\end{equation}
where $\tau \geq \lambda_\mathrm{max} (A^TA) = \sigma^2_\mathrm{max}(A)$. With this approximation, the second step of \eqref{eqn:alt} can be solved iteratively using soft-thresholding as derived below:
\begin{IEEEeqnarray}{rCl}
& & \arg \min_\x \, \|\x\|_1+ \langle \bl_k, \y - A\x \rangle + \frac{\mu_k}{2} \left(\|\y - A\x_k\|_2^2 + \langle 2A^T(A\x_k-\y), \x - \x_k \rangle + \tau\|\x-\x_k\|_2^2 \right) \IEEEeqnarraynumspace \\
& = & \arg \min_\x \, \frac{2}{\mu_k}\|\x\|_1 + \left \langle 2A^T\left(A\x_k - \y - \frac{1}{\mu_k}\bl_k\right), \x \right \rangle + \tau\|\x-\x_k\|_2^2 \\
& = & \arg \min_\x \, \frac{2}{\mu_k}\|\x\|_1 - \left \langle 2\tau \x_k - 2A^T\left(A\x_k - \y - \frac{1}{\mu_k}\bl_k\right), \x \right \rangle + \tau\|\x\|_2^2 \\
& = & \arg \min_\x \, \frac{1}{\mu_k\tau}\|\x\|_1 - \left \langle \x_k - \frac{1}{\tau}A^T\left(A\x_k - \y - \frac{1}{\mu_k}\bl_k\right), \x \right \rangle + \frac{1}{2}\|\x\|_2^2 \\
& = & \arg \min_\x \, \frac{1}{\mu_k\tau}\|\x\|_1 + \frac{1}{2}\left \| \x - \left(\x_k - \frac{1}{\tau}A^T\left(A\x_k - \y - \frac{1}{\mu_k}\bl_k\right) \right)\right \|_2^2 \\
& = & \shrink\left(\x_k - \frac{1}{\tau}A^T\left(A\x_k - \y - \frac{1}{\mu_k}\bl_k\right), \frac{1}{\mu_k\tau}\right)
\end{IEEEeqnarray}
\smallbreak
Thus, the iteration to solve the second step of \eqref{eqn:alt} can be written as:
\begin{equation}
\x_{l+1}  =  \shrink\left(\x_l - \frac{1}{\tau}A^T\left(A\x_l + - \y - \frac{1}{\mu_k}\bl_k\right), \frac{1}{\mu_k\tau}\right).
\end{equation}
Thus, the entire algorithm can be summarized as Algorithm \ref{alg:exact}.
\begin{algorithm}[h]
\caption{Exact ALM}
\begin{algorithmic}
\WHILE{not converged}
\WHILE{not converged}
\STATE $t_1 \leftarrow 1$, $\z_1 \leftarrow \x_j$
\WHILE{not converged}
\STATE $\x_{l+1}  =  \shrink\left(\z_l - \frac{1}{\tau}A^T\left(A\z_l - \y - \frac{1}{\mu_k}\bl_k\right), \frac{1}{\mu_k\tau}\right)$
\STATE $t_{l+1} = 0.5\left( 1 + \sqrt{1+4t_l^2}\right)$
\STATE $\z_{l+1} = \x_{l+1} + \frac{t_l - 1}{t_{l+1}}(\x_{l+1} - \x_l)$
\ENDWHILE
\ENDWHILE
\STATE $\bl_{k+1} = \bl_k + \mu_k (\y - A\x_{k+1})$
\STATE $\mu_{k+1} = \rho\cdot\mu_k$
\ENDWHILE
\end{algorithmic}
\label{alg:exact}
\end{algorithm}



%\section{L1 Minimization via Subgradient Descent} % TODO
{\bf Computing the steepest descent direction for rectangular integration }
%First order methods for solving extremely tall $\ell_1$ minimization problems are limited in speed primarily
%by their memory bandwidth requirements.  Furthermore, they require tuning and proper scaling of the data
%to ensure that they converge to the correct solution, and that the convergence occurs in a small number
%of iterations.  For this reason, these notes investigate algorithms that exhibit the following properties:
\begin{eqnarray}
\argmin{\x\in \Re^n} ||A \x - \bb||_1 \\
A \in \Re^{m \times n}\\
b \in \Re^m \\
\e = A \x - \bb \in \Re^m\\
f = ||\e||_1 \in \Re \\
\a_i = A(i,:) \in \Re^{n}\\
\end{eqnarray}

For simplicity we make the following assumptions on the problem matrices $A$ a $\bb$.
\begin{itemize}
\item $A$ and $\b$ are general arrays, i.e. we do not (yet) analyze special cases
where $A$ drops rank, there are duplicate rows, zero rows, etc.
\item $m$ is larger than $n$ to a degree that a $m \times n$ matrix-vector multiplication (gemv)
takes more time than inverting an $n\times n$ matrix.  
\end{itemize}

For some arbitrary point $\tilde\x$, we wish to compute a steepest descent
direction of $f$ and $\tilde\x$.  We begin by analyzing two cases.

In the first case, $\x$ does not lie on any attracting hyperplanes, i.e. $\e_i(\tilde\x)=0
\forall i$. In this case, the function is locally linear, and the gradient can be computed
by applying the appropriate sign change to each row of $A$, and then summing the rows of $A$:
\begin{eqnarray}
f_{local} &=& \sum_i |e_i(\tilde\x)| = \sign(\e(\tilde\x)) .* e(\tilde\x) \\
\nabla_x f_{local} &=& A^T * \sign(\e(\tilde\x)) \\
d &=& -A^T * \sign(\e(\tilde\x))
\end{eqnarray}
Note that once $\tilde\x$ is known, $d$ can be computed in a single pass through $A$.

The second special case analyzed is for the case where $\tilde\x$ lies on the intersection
of $n$ intersecting hyperplanes.  Without loss of generality, we reorder and partition the
matrices $A$ and $\bb$ such that the top $n$ rows correspond to the equations that are solved
with equality at $\tilde\x$:
\begin{eqnarray}
A = \left[\begin{array}{c}A_1 \\ A_2 \end{array}\right]\\
\bb = \left[\begin{array}{c}\bb_1 \\ \bb_2 \end{array}\right]\\
A_1 \in \Re^{n \times n} \\
A_2 \in \Re^{m-n \times n}
\end{eqnarray}
At $\tilde\x$, $\bb_1 = 0$.  Under our assumptions on $A$, $A_1$ is invertible.
We make the following state-dependent affine coordinate transformation:
\begin{eqnarray}
\hat\x &=& A_1\x - \bb_1 \\
\x &=& A_1^{-1} (\hat\x + \bb_1)
\end{eqnarray}
With this substitution,
\begin{eqnarray}
f(\hat\x) &=& ||\hat\x||_1 + ||A_2 A_1^{-1} (\hat\x + \bb_1) - \bb_2 ||_1 \\
&=& ||\hat\x||_1 + || (A_2 A_1^{-1}) \hat\x + (A_2 A_1^{-1} \bb_1 - \bb_2) ||_1 \\
&=& ||\hat\x||_1 + || \hat A \hat\x + \hat \bb ||_1
\end{eqnarray}
For infinitesimal $\hat\x$, the signs of the entries of the second term of $f$ will be
governed by the signs of the entries of $ \hat \bb = A_2 A_1^{-1} \bb_1 - \bb_2$.  
%Note: $A_1^{-1} \bb_1$ is the center of the linear
Note that due to our reordering of the rows, the entries of $\hat \bb$ are non-zero,
and the sign is well defined.
Therefore, $f$ locally reduces to:
\begin{eqnarray}
f(\hat\x) &=& ||\hat\x||_1 + \sign(\hat \bb)^T(\hat A \hat \x + \hat \bb) \\
&=& ||\hat\x||_1 + (\sign(\hat \bb)^T \hat A) \hat \x + (\sign(\hat \bb)^T  \hat \bb) \\
&=& ||\hat\x||_1 + \g^T \hat \x + ||\hat \bb||_1
\end{eqnarray}
For the purposes of deriving a descent direction, we can neglect the constant last term, leaving:
\begin{eqnarray}
f &=& ||\hat\x||_1 + \g^T \hat \x
\end{eqnarray}
This function has a stepest descent direction:
\begin{eqnarray}
\hat d &=& -\shrink(\g)
\end{eqnarray}
where the operator $\shrink(\x) = \sign(\x).*\max(|\x|-1,0)$ is the standard
soft thresholding operator. Mapping back to the original coordinate system and unrolling substitutions,
we have:
\begin{eqnarray}
\d &=& A_1^{-1} \hat \d \\
 &=& -A_1^{-1} \shrink(\g) \\
 &=& -A_1^{-1} \shrink(\hat A ^T \ \sign(\hat \bb)) \\
 &=& -A_1^{-1} \shrink(\hat A ^T \ \sign(A_2 A_1^{-1} \bb_1 - \bb_2)) \\
 &=& -A_1^{-1} \shrink(A_1^{-T}A_2^T \sign(A_2 A_1^{-1} \bb_1 - \bb_2))
\end{eqnarray}
Unfortunately, while $\hat \d$ is the steepest descent direction for the optimization in transformed coordinates,
$\d$ is in general not the steepest descent direction for the original problem.  
{\em What if only some rows of $e_1$ are zero?}


%\input{sec_alm_allen}
%\input{sec_trapezoid}

\section{Conclusion}


