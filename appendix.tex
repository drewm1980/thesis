\section{L1 Minimization via Interior Point Method} % TODO

\section{L1 Minimization via Subgradient Descent} % TODO

\section{L1 Minimization via Augmented Lagrange Multiplier} % TODO add more complete derivation of ALM
 
In this section we discuss the computational issues related to the
implementation of Algorithm 1, which is repeated here for convenience. It is
not hard to see that its computational complexity is dominated by the two steps
where the $\ell^1$-norm minimization problems are solved; namely Step 6 for
iterative registration, and Step 14 for global sparse representation.
Fortunately, many fast algorithms for solving these problems have been proposed
over the past ten years. We refer the interested reader to \cite{YangA2010-pp}
for a more comprehensive survey of the developments in this area.  That work
suggests that \emph{Augmented Lagrange Multiplier} (ALM) algorithms
\cite{Bertsekas1982} strike a good balance between scalability and accuracy: as
first order methods, they require only lightweight vector operations and
matrix-vector multiplications at each iteration, making them preferable to more
classical solutions such as interior point methods.  However, compared to other
first-order methods, they achieve higher accuracy with a fixed computational
budget.

We use Step 14 as an example to illustrate the ALM
method, since solving Step 6 is very similar. Recall that in
Step 14 the problem we are interested in is:
\begin{equation}
\min_{\x, \e} \| \x \|_1 + \|\e\|_1 \quad \subj \quad \y =
A \x + \e.
\end{equation}
Its corresponding augmented Lagrangian function is
\begin{equation}
L_\mu (\x,\e,\blamda) = \|\x\|_1 + \|\e\|_1 + \langle \blamda, \y-A\x - \e \rangle + \frac{\mu}{2} \|\y - A\x - \e\|_2^2,
\end{equation}
where $\blamda$ is the Lagrange multiplier and $\mu > 0$ is a
penalty parameter. The ALM method seeks a saddlepoint of $L_\mu
(\x,\e,\blamda)$ by alternating between optimizing with respect
to the primal variables $\x, \e$ and updating the dual variable
$\blamda$, with the other fixed, as follows:
\begin{equation}
\left \{
\begin{array}{lll}
(\x_{k+1},\e_{k+1})  =  \arg\min_{(\x,\e)} \, L_{\mu} (\x,\e,\blamda_k),\\
\blamda_{k+1}  =  \blamda_k + \mu (\y - A\x_{k+1} - \e_{k+1}). \\
\end{array}
\right .
\label{eq:alm}
\end{equation}
Although updating $\blamda$ is trivial,
minimizing $L_{\mu} (\x,\e,\blamda_k)$ with respect to both
$\x$ and $\e$ could still be costly. To further reduce the
complexity of the problem, we adopt an approach used in
\cite{YangJ2009-pp}, called \emph{alternating direction
method of multipliers} (ADM) \cite{Glowinski1975-TR}, which alternates between minimizing $L_{\mu} (\x,\e,\blamda_k)$
over $\x$ (with $\e$ fixed) and over $\e$ (with $\x$ fixed). After solving these two subproblems, the Lagrange multiplier $\blamda$ is updated, yielding an iteration of the form:
\begin{equation}
\left \{
\begin{array}{lll}
\e_{k+1}  =  \arg\min_{\e} \, L_{\mu} (\x_k,\e,\blamda_k),\\
\x_{k+1}  =  \arg\min_{\x} \, L_{\mu} (\x,\e_{k+1},\blamda_k),\\
\blamda_{k+1}  =  \blamda_k + \mu (\y - A\x_{k+1} - \e_{k+1}). \\
\end{array}
\right .
\label{eq:adm}
\end{equation}
As the objective function is convex and alternation is between two
terms, this procedure is guaranteed to converge to a global optimum (see \cite{YangJ2009-pp} and references therein).

In order to discuss the solution to the above subproblems, we
need to define the following soft-thresholding operator for a
vector $\x$ and a scalar $\alpha \geq 0$:
\begin{equation}
\mathcal{T}(\x,\alpha) = \textup{sign}(\x)\cdot \max \{|\x| - \alpha, 0\},
\end{equation}
where all the operations are performed component-wise. It is
easy to show that the subproblem with respect to $\e$ has a
closed-form solution given by the soft-thresholding operator:
\begin{equation}
\e_{k+1} = \mathcal{T}(\y - A\x_k + \mu^{-1}\blamda_k, \mu^{-1}).
\end{equation}
To solve the subproblem associated with $\x$, we
apply a first-order $\ell^1$-minimization method,
called \emph{fast iterative shrinkage-threshold algorithm}
(FISTA) \cite{BeckA2009}. The main idea of FISTA is to
iteratively minimize a quadratic approximation $Q(\x, \z)$ to
$L_{\mu} (\x,\e_{k+1},\blamda_k)$ around a point $\z$, which is
carefully chosen in order to achieve a good convergence
rate. We summarize the entire ALM
algorithm as Algorithm~\ref{alg:alm}, where $\gamma$ denotes the
largest eigenvalue of the matrix $A^TA$. For the choice of parameter $\mu$, we take the same strategy as
in \cite{YangJ2009-pp} and set $\mu = 2m / \|\y\|_1$.
\begin{algorithm}[t]
\caption{\bf (Augmented Lagrange Multiplier Method for Global
Recognition)}
\begin{algorithmic}[1]
\begin{small}
\STATE {\bf Input:} $\y \in \Re^m$, $A \in \Re^{m \times n}$,
$\x_1 = \mathbf{0}$, $\e_1 = \y$, $\blamda_1 =
\mathbf{0}$.
\WHILE{not converged ($k = 1,2,\ldots$)}
\STATE $\e_{k+1} = \mathcal{T}\left(\y - A\x_k +
\frac{1}{\mu}\blamda_k, \frac{1}{\mu}\right)$;
\STATE $t_1\leftarrow 1$, $\z_1 \leftarrow \x_k$, $\w_1 \leftarrow \x_k$;
\WHILE{not converged ($l = 1,2,\ldots$)}
\STATE $\w_{l+1} \leftarrow \mathcal{T}\left(\z_l +
\frac{1}{\gamma}A^T\left(\y - A\z_l - \e_{k+1} +
\frac{1}{\mu}\blamda_k\right), \frac{1}{\mu\gamma}\right)$;
\STATE $t_{l+1} \leftarrow \frac{1}{2}\left( 1 +
\sqrt{1+4t_l^2}\right)$;
\STATE $\z_{l+1} \leftarrow \w_{l+1} + \frac{t_l - 1}{t_{l+1}}(\w_{l+1} - \w_l)$;
\ENDWHILE
\STATE $\x_{k+1} \leftarrow \w_{l}$,  \; $\blamda_{k+1} \leftarrow \blamda_k + \mu (\y - A\x_{k+1} - \e_{k+1})$;
\ENDWHILE \STATE
{\bf Output:} $\x^* \leftarrow \x_k, \e^* \leftarrow \e_k$.
\end{small}
\end{algorithmic}
\label{alg:alm}
%\vspace{-4mm}
\end{algorithm}

We have selected this algorithm because it strikes the best
balance between speed, accuracy, and scalability for our problem out of
many algorithms that we have tested. We refer the interested reader to
\cite{YangA2010-pp} for a more detailed discussion of competing
approaches.  On a Mac Pro with
Dual-Core 2.66GHz Xeon processors and 4GB memory,
running on our database containing images size $80\times 60$
pixels from 109 subjects under 38 illuminations,
our C implementation of Algorithm 1 takes
about 0.60 seconds per subject for alignment and about 2.0
seconds for global recognition. Compared to the highly
customized interior point method used in the conference version
of this paper \cite{Wagner2009-CVPR}, this new algorithm is
only slightly faster for per subject alignment. However, it is
much simpler to implement and it achieves a
\emph{speedup of more than a factor of 10} for global
recognition!
